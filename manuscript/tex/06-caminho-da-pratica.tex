\chapterPhotoTwoPageLeft{sitting-lake-Gr-crop}

\chapterPhotoTwoPageRight{sitting-lake-Gr-crop}

\setlength{\chapterTitleTopSkip}{0mm}
\definecolor{photoChapterText}{gray}{1}

\chapter{O Caminho da Prática}

\clearpage

\quoteTitleFmt{Dāna (Dádiva)}

\begin{verse}

Bhikkhus, se os seres soubessem, como eu sei, o resultado de dar e
partilhar, não comeriam sem terem dado, nem consentiriam a nódoa da
mesquinhez a obcecá"-los e a enraizar"-se em suas mentes. Mesmo que fosse
a sua derradeira colherada, a sua última porção, não comeriam sem o
partilharem, assim houvesse alguém com quem o~fazer.

{\raggedleft
Itivutttaka, 18
\par}

\end{verse}


\section{Onde começa o caminho do Budismo?}

O caminho budista procura eliminar o sofrimento e a toxidade mental que
é a sua causa. O mais grosseiro dos agentes subversivos mentais é o
apego egoísta aos bens materiais. Por este motivo, o caminho budista
começa com o cultivo da generosidade. Ao desenvolver o hábito da dádiva
e da partilha, a mente limpa a sua forma limitada e invejosa de se
apropriar das coisas. O acto de dar requer que, quem dá, tenha em conta
as necessidades dos outros, gerando por sua vez empatia. A prática da
dádiva cria alegria ao dador e aumenta o sentido de afecto e de apoio
mútuo nas famílias e nas comunidades.

\section{O que é que determina a característica espiritual da dádiva?}

A intenção é a chave para todas as práticas budistas. Um acto de
generosidade motivado pelo desejo de recompensa, quer seja de proveito
material, quer de benefícios mais intangíveis -- estatuto, reputação,
respeito ou amor - tem pouco poder na purificação da mente. Tal dádiva
é pouco mais que uma forma de troca. Dar sem expectativas é um acto
poderoso, reduz os apegos e expande o auto"-respeito e a alegria.

\section{O que é considerado mais valioso: oferecer aos mosteiros ou a
  instituições de caridade?}

Um dos maiores discípulos leigos de Buda, Anāthapindika, é visto como um
modelo para os budistas leigos. Tornou"-se famoso por oferecer esmola
diária tanto à ordem monástica como aos pobres e necessitados. Os
budistas são encorajados a apoiar as comunidades monásticas locais, mas
não descuidando os deveres para com todos os que sofrem e que têm
necessidades de assistência.

\clearpage
\quoteTitleFmt{Sīla (Moralidade)}

\begin{verse}

Além disso, deveriam lembrar"-se do vosso comportamento virtuoso
como inquebrantável, perfeito, inconspurcável, imaculado, libertador,
louvado pelos sábios, não capturável, conducente à concentração. Sempre
que um discípulo dos seres nobres se recorda das virtudes, a sua mente
não se deixa subjugar por paixões, nem por aversões, nem por ilusões. A
mente torna"-se então perfeitamente direccionada, baseada na virtude. E
quando a mente está perfeitamente direccionada, o discípulo dos seres
nobres adquire um sentido do propósito, obtém um sentido do Dhamma,
ganha alegria na ligação ao Dhamma. Naquele que está feliz, surge o
êxtase. Naquele que está enlevado, o corpo acalma. Aquele cujo corpo
está sereno, experimenta o relaxamento. Naquele que está em paz, a mente
concentra"-se.

{\raggedleft
Aṅguttara Nikāya, Livro dos Três 70
\par}

\end{verse}

\section{Existe alguma característica"-chave da atitude budista relativamente à
  moralidade?}

Sim, a ênfase na volição. O Buda disse que a moralidade é volição. Um
acto é definido como moral ou imoral, de acordo com os factores mentais
presentes na concretização do acto. Um acto levado a cabo com um estado
mental nocivo tem, automaticamente, mau kamma; a natureza específica do
acto -- a justificação para tal -- só afecta a severidade do kamma
criado. A prática da consciência presente no momento das intenções
pessoais e o desenvolvimento dos estados mentais que permitem, a cada
um, refrear"-se de intenção maldosas, são características vitais na
moralidade budista.

\section{Quais são os cinco preceitos?}

Os cinco preceitos constituem a base do código moral no Budismo.
Consistem na determinação de se abster de:

\begin{enumerate}
\item Tirar a vida
\item Roubar e enganar
\item Ter uma conduta sexual inapropriada
\item Mentir
\item Ingerir álcool e qualquer substância narcótica
\end{enumerate}

Quase todas as cerimónias presididas pelos membros do Sangha incluem uma
passagem, na qual os budistas leigos pedem formalmente os cinco
preceitos ao monge sénior. O monge recita os preceitos, um de cada vez,
e o leigo budista repete"-os a seguir. O parafrasear dos preceitos é
instrutivo: `Eu assumo o compromisso de me abster de tirar a vida
(roubar e enganar, etc.) como um meio de educar a minha conduta'.

\section{Quais são as semelhanças e as diferenças entre o código moral budista,
  e os das outras principais tradições religiosas do mundo?}

As acções a que se referem os cinco preceitos são abordadas em todos os
códigos morais mais importantes no mundo, e muito provavelmente da mesma
maneira. (O tipo de vida a ser respeitado, ou as definições de má
conduta sexual, por exemplo, variam de religião para religião.) A única
característica da moralidade budista é a de que, mais do que ser
percebido como um assunto de obediência a uma lista de mandamentos
ditados por uma divindade, é vista como um tipo de treino, ou educação
da conduta. Só quando os preceitos são compreendidos desta forma, e
assumidos voluntariamente, é que proporcionam uma fundação para a
prática mais avançado da mente defendida pelo Buda.

Sob a perspectiva budista, as acções imorais produzem resultados, tão
naturalmente como consistentemente, como as consequências que advêm de
se pôr a mão no fogo. Tal como a maioria das pessoas que vê o sofrimento
desta última forma, não como uma punição divina, mas como uma
consequência natural do fogo, da pele e da falta de juízo da pessoa que
põe em contacto uma com a outra, também o Budismo compreende o
sofrimento que surge das acções que prejudicam o responsável e os
outros.

Um dos maiores desafios enfrentados pelas sociedades humanas é o de
encontrar meios que promovam famílias e comunidades que se baseiem na
confiança e respeito mútuo, e nas quais todos os membros se sintam
seguros e valorizados. O Buda ensinou que a abstenção voluntária de
acções e palavras negativas tem um papel muito importante neste
processo.

\section{Existe alguma justificação para a violência e para o crime?}

A resposta breve para esta pergunta é: não. Neste ponto, os ensinamentos
budistas são inabalavelmente claros. Qualquer que seja a justificação
que exista para se matar, se a determinação de matar estiver presente na
mente do assassino, já se criou kamma negativo, o que levará a
consequências nefastas. As razões para matar apenas determinam a
severidade do kamma criado. Por exemplo, o assassínio premeditado de um
benfeitor, por vontade, ou ódio, criará muito mais kamma do que matar
um inimigo para proteger a família, o país ou a religião. Não existem
quaisquer casos nos quais o Buda tenha defendido a violência, nem sequer
como último recurso. Num verso famoso o Buda afirmou:

\begin{verse}
  Neste mundo o ódio nunca acaba através do ódio;\\
  Mas cessará através do não-ódio.\\
  Esta é uma lei eterna.

  Dhammapada 5
\end{verse}

\section{Os países budistas são totalmente pacifistas?}

Os países predominantemente budistas reconhecem -- como o fazem todos os
países -- a necessidade de defender o interesse nacional. Enquanto as
guerras motivadas por cobiça, inveja ou ódio, são claramente vistas como
imorais, a maioria dos budistas consideram a guerra de auto defesa como
um mal necessário. Contudo, reconhecem que os envolvidos na luta de tais
guerras não se livrarão de mau kamma, uma vez que o kamma ocorre
inevitavelmente com a intenção de matar, independentemente da razão que
o tenha compelido a tal. Nesta perspectiva, o heroísmo dos membros das
forças armadas reside no acto voluntário de criar mau kamma e
subsequente futuro sofrimento, a bem da nação.

\section{Qual a relação que existe entre manter os preceitos e a prática da
  meditação?}

O progresso na via do Buda para despertar só é possível quando existe
harmonia entre o interior e o exterior. Se quem medita se permite que
as acções e palavras sejam influenciadas pelo estados mentais
perniciosos, acaba por estar a fortalecer os mesmos hábitos que procura
abandonar durante a meditação. A falta de manutenção dos preceitos é a
maior causa para a auto"-aversão, culpa e ansiedade. Cria problemas nas
relações, tornando a vida desgastante e complicada. A preservação dos
preceitos ajuda a manter um ambiente seguro e estável, conducente à
prática do Dhamma. Preservar os preceitos liberta a mente do remorso,
incutindo um sentido de respeito e bem"-estar geral, preparando"-a para
futuros progressos no caminho.

\clearpage\thispagestyle{empty}\mbox{}%
\photoFullBleed{buddha-back-Gr-crop.jpg}

\clearpage
\quoteTitleFmt{Bhāvana (Cultivo mental)}

\begin{verse}
  Existem estas raízes de árvores, que são cabanas vazias. Meditai, monges, não
  vos detenháis, ou vos arrependereis disso, mais tarde. Esta é a nossa
  instrução, para vós.

  Majjhima Nikāya 19
\end{verse}

\section{Porquê meditar?}

Os seres humanos querem ser felizes e não querem sofrer. A meditação é o
meio mais eficaz de cultivar as causas internas da felicidade e de
erradicar as causas internas do sofrimento.

A meditação traz enormes benefícios físicos. As novas tecnologias, tal
como os fMRI, revelaram que a meditação regular ao longo de muitos anos
tem um efeito positivo, tanto na função, como na estrutura mental. A
meditação reduz a tensão e, ao fazê"-lo, reforça o sistema imunitário,
levando à diminuição de doenças e da severidade das mesmas. O treino da
mente desenvolve a capacidade de libertar os estados negativos mentais,
reduzindo, assim, os factores psicossomáticos envolvidos na doença
física. Ao manter a capacidade de acalmar a mente, quem medita é mais
capaz de lidar favoravelmente com sentimentos depressivos, ansiedade e
medos, que frequentemente acompanham a doença. Tais competências reduzem
o sofrimento mental que acompanha as doenças físicas e aceleram o
processo de cura. No fim da vida, os praticantes de meditação
experientes são capazes de deixar este mundo em paz.

A primeira tarefa para quem medita é aprender a manter a atenção num
objecto. Ao fazê"-lo, fica exposto o comportamento normal e destreinado
da mente, podendo aprender a identificar e a lidar com os estados
mentais dispersos e confusos, e a alimentar os saudáveis. Uma
competência valiosa, aprendida neste estádio de meditação, é o controlo
dos impulsos, um dos mais significativos prognósticos do sucesso em
qualquer das caminhadas da vida. A calma, e o sentido de bem"-estar que
advêm da meditação, levam a uma auto"-suficiência interior. Como
resultado, o impulso para buscar prazer através dos sentidos é reduzido
em muito, e os comportamentos prejudiciais, tais como o uso de drogas,
são abandonados, sem qualquer pesar. Pensamentos nobres de generosidade
e de bondade surgem naturalmente na mente, e com uma frequência
acrescida.

A mente bem treinada pela meditação possui clareza e força suficientes
para se aperceber da verdadeira natureza das coisas, usando a
experiência directa. Ver as coisas a esta luz, permite que sejam
libertadas as assunções e os apegos equívocos, que são as causas"-raiz do
sofrimento humano. Por fim, a meditação conduz ao despertar e à
libertação total do sofrimento e de suas causas, e a uma mente pura e
desimpedida nas suas funções, repleta de sabedoria e compaixão.

\section{Qual é a melhor altura para meditar?}

Um grande mestre tailandês disse: `Se tens tempo para respirar, tens
tempo para meditar.' Dito isso, muitas pessoas consideram que a
meditação no princípio da manhã é o período mais produtivo para uma
sessão de meditação formal. O corpo está descansado e a mente
razoavelmente liberta de sua habitual ocupação. Uma sessão de meditação
é uma forma maravilhosa de começar um novo dia. Quem medita, ao observar
os efeitos positivos no estado mental, ao longo do dia, (particularmente
nas horas que se seguem à sessão de meditação), desenvolve uma grande
confiança no valor que a meditação tem nas suas vidas.

\section{Qual é o método básico para meditar?}

Embora certos princípios fundamentais inspirem todas as formas de
meditação budista, existe uma grande variedade de técnicas específicas
usadas na sua expressão. Não existe só um método básico de meditação,
existem muitos. Uma das abordagens particulares é a que se segue.

Primeiro, dá"-se atenção às condições externas. É vantajoso existir um
espaço específico criado à parte para se praticar a meditação.
Deve"-se usar roupa solta e assegurar ventilação razoável: as divisões
abafadas induzem ao entorpecimento.

A melhor postura é a de sentar"-se de pernas cruzadas, da qual se obtém a
sensação de estabilidade e de auto confiança como suporte à prática da
meditação. A maioria das pessoas sentem que é útil usar uma pequena
almofada para apoiar a zona lombar. A postura deve ser direita, mas não
rígida; quem medita busca um equilíbrio entre o esforço e a
descompressão (o fluir solto da respiração é um sinal de tal ter sido
atingido). Quem medita coloca as mãos no colo, ou nos joelhos, e fecha
delicadamente os olhos (podem manter"-se ligeiramente abertos e sem se
focarem, caso surja o risco de adormecer). Se não for possível estar
sentado de pernas cruzadas, senta"-se num assento, mas se possível sem
se apoiar nas costas.

Quem medita começa por alguns minutos de reflexão, relembrando a sua
motivação, aplicando a técnica e evitando as armadilhas. A seguir,
dirige sistematicamente a atenção da cabeça para os pés, identificando e
relaxando qualquer tensão no corpo. Ao descobrir algum nó de tensão, por
exemplo nos ombros, deve aumentar, conscientemente, a tensão por um
segundo ou dois, e depois descomprimir. Fisicamente preparado, deve"-se
focar agora no objecto específico de meditação escolhido. Podemos passar
a apresentar a meditação na respiração, a forma mais popular da tradição
budista: quem medita, treina a presença na sensação de respirar num
ponto do corpo onde a sinta mais claramente. Para a maioria das pessoas
este ponto fica na área que é a ponta do nariz. Não é adequado forçar a
respiração, sob que forma for. Deve"-se ficar meramente consciente da
presença da sensação consoante vai aparecendo.

Para manter a atenção na respiração, pode"-se recitar mentalmente um
mantra bissílabo, a primeira sílaba na inspiração, e a
segunda na expiração. A palavra mais usada nos budistas tailandeses é
Bud"-dho, mas recitar `in', na inspiração, e `out' na expiração,
também resulta. Contar as respirações também pode ser usado para manter
a ligação entre a mente e a respiração. A forma mais simples de contar a
respiração é contar em ciclos de dez, contando 1 para cada inspiração e
expiração, continuando até 10, e depois recomeçando no~1.

Qualquer que seja a técnica adoptada, a mente vai vaguear. Tal como
acontece quando se aprende a tocar um instrumento musical, ou uma língua
estrangeira, quem medita deve ser paciente e dedicado, e deve acreditar
que a meditação a longo prazo vale o tempo e o esforço. Gradualmente, a
mente se acalmará.

\section{Qual a finalidade da meditação a andar e como se pratica?}

A meditação a andar proporciona tanto um suplemento, como uma
alternativa à meditação sentada. Algumas das pessoas que meditam
preferem"-na à sentada e podem fazer dela a sua prática principal. A
meditação a andar é uma opção particularmente útil quando a doença, a
fadiga, ou um estômago muito cheio, representam uma grande dificuldade
em estar sentado. Enquanto na meditação sentada, se desenvolve a
consciência estando sossegado, na meditação a andar desenvolve"-se usando
o movimento. Por conseguinte, praticar a meditação a andar combinando"-a
com a sentada, ajuda a desenvolver uma concentração flexível em geral,
que pode ser integrada mais facilmente no dia"-a-dia, do que a que é
desenvolvida apenas pela meditação sentada. E como bónus acrescido, a
meditação a andar é um bom exercício.

Para praticar a meditação a andar, determina"-se um espaço de caminhada
com a distância de 20 ou 30 passos, marcando um ponto no meio. O
praticante começa num dos extremos do caminho, com as mãos entrelaçadas
à sua frente{.} A seguir começa a andar em direcção ao outro extremo,
onde faz uma breve paragem, antes de se virar e regressar ao ponto de
partida. Após outra breve paragem, repete isto, anda para a frente e
para trás nesse espaço, ao longo de toda a sessão de meditação. O
praticante usa o começo, o fim, e o meio do caminho como pontos de
inspecção que asseguram não se ter distraído. A velocidade usada para
andar varia de acordo com o estilo de meditação a praticar e com a
preferência individual.

No esforço inicial, pode"-se usar uma variedade de métodos para
conseguir transcender os cinco entraves à meditação. Um método muito
usado, semelhante àquele proposto na meditação sentada, é o de usar as
palavras de duas sílabas na meditação (mantra): pé direito a tocar o
chão e a recitar mentalmente a primeira sílaba; pé esquerdo a tocar o
chão e a recitar mentalmente a segunda sílaba. Como alternativa, pode"-se
colocar a atenção nas sensações das solas dos pés consoante tocam o
chão. Tal como na meditação sentada, a intenção é a de usar o objecto de
meditação como um meio de cultivar suficiente concentração, estado de
alerta e esforço, para que a mente fique fora do alcance dos
obstáculos, criando, assim, as melhores condições para a contemplação da
natureza do corpo e da mente.

\section{Quais são as principais obstruções à meditação?}

Quem medita confronta"-se com um número de hábitos mentais profundamente
enraizados, que impedem que a mente experimente a paz interior e
mantenha o insight. O Buda mencionou cinco em particular. Qualquer que
seja a técnica de meditação usada, a primeira tarefa é a de ultrapassar
estas quatro interferências (\emph{nīvaraṇa}):

\begin{enumerate}
\item
  O primeiro obstáculo à meditação é o deleite retirado de formas
  visíveis, sons, odores, sabores, e sensações físicas. Ao sentar"-se
  para meditar, as memórias e a imaginação baseadas em assuntos que
  agradam ao praticante, enredam as mentes distanciando"-o do trabalho
  que tem em mãos. A expressão mais poderosa destes entraves é a
  fantasia sexual, mas também pode aparecer numa imparável sequência de
  pensamentos ligados à comida, ao entretenimento, aos desportos, aos
  assuntos políticos -- a qualquer coisa que seja agradável pensar.
\item
  O segundo entrave é a má vontade. Na meditação, a má vontade varia
  desde intensos sentimentos de ódio e preconceito, num extremo do
  espectro, até ao outro extremo, num desvio subtil de experiências
  sentidas como desagradáveis. A má vontade pode ser focada em si
  próprio, nos outros ou no ambiente envolvente.
\item
  O terceiro obstáculo é a preguiça, o entorpecimento. Este entrave
  inclui a preguiça, o enfado, e a mente bloqueada. Na sua pior
  manifestação, o praticante adormece, ou entra num obscuro estado
  desprovido de clareza. Nas formas mais subtis, o obstáculo é sentido
  como uma leve rigidez, ou como uma ausência de vitalidade mental.
\item
  O quarto obstáculo tem dois aspectos. O primeiro é a agitação mental -
  a mente do macaco -- na qual a mente pula incansavelmente de um
  pensamento para outro, tal como um macaco a saltar de ramo em ramo,
  sem qualquer objectivo definido. O segundo, é a complacência nas
  preocupações, ansiedades e culpas.
\item
  O quinto entrave é a oscilação mental e a indecisão, uma forma de
  dúvida. A dúvida é racional, quando se reconhece haver falta de
  informação necessária para tomar uma boa decisão. A dúvida torna"-se um
  obstáculo, quando quem medita tem toda a informação necessária para
  prosseguir, mas não se compromete com um caminho de prática. O
  impedimento faz perguntas do tipo: `E se não der certo?', `E se
  for uma perda de tempo?', e fica à espera das respostas em vez de se
  esforçar naquilo que lhe daria essas mesmas respostas.
\end{enumerate}

\section{Qual deverá ser a duração de uma sessão de meditação?}

Ao princípio, os praticantes não deverão insistir em ficar sentados por
períodos de tempo maiores do que os que estejam preparados para fazer.
Esse excesso de esforço inicial pode levar a uma reacção abaixo da linha
que resulta com eles, fazendo"-os desistir. É preferível começar com
cinco minutos, aumentando gradualmente para uma meia hora, e depois,
pouco a pouco, alcançando quarenta e cinco minutos, ou uma hora. Quem é
experiente consegue sentar"-se durante duas horas, ou até três, mas mais
do que a quantidade do tempo, o mais importante é a qualidade do tempo.

\section{O que é estar plenamente consciente?}

A forma mais simples para definir o estado consciente é não se
esquecer. \emph{Sati} é a faculdade mental que transporta (algo) à mente
e que fica na mente. Se, em determinada situação, se trouxer à mente
tudo quanto é preciso lembrar, e não se distrair, isso é \emph{sati}. É
crucial incluir na mente a manutenção das dimensões morais de cada
acção pessoal: um arrombador pode saber como se focar na tarefa naquele
momento, mas não possui \emph{sati}. Na meditação, \emph{sati}
manifesta"-se como a consciência do objecto.

\emph{Sati} deve ser acompanhada pelo estado de alerta e esforço
apropriado. Os adeptos da prática de \emph{sati} estão conscientes do
corpo, das emoções, dos sentimentos, pensamentos e sentidos, sem se
identificarem com eles. Sabem como proteger a mente de estados confusos,
e como lidar com esses estados, uma vez que já tenham acontecido. Sabem
criar estados mentais benéficos, e sabem manter aqueles que já tenham
surgido.

\section{Como se ultrapassam as obstruções durante a meditação?}

A primeira estratégia usada para lidar com as obstruções é ser sincero.
Tão"-somente reconhecer que um obstáculo é um obstáculo, não se
interessando por ele e conscientemente regressando ao objecto da
meditação, sem se consentir pensamentos de desencorajamento, frustração
ou de desapontamento. Ao regressarem ao objecto de meditação vezes sem
fim, os praticantes desenvolvem a capacidade de manter a consciência no
momento presente. Como resultado, as obstruções acontecem com menos
frequência, são identificadas mais rapidamente e abandonadas mais
facilmente.

No caso de um impedimento particular ter atingido um nível crónico, e
não responder a uma técnica simples de retirada de interesse e regresso
ao objecto, são aplicados antídotos específicos. Reflectir nos aspectos
pouco atractivos do corpo físico, permite opor"-se a, e minar, a
intoxicação dos aspectos atractivos. Gerar pensamentos de amor e
compaixão opõe"-se a, e mina, pensamentos entrincheirados na má vontade.
Uma sábia reflexão sobre a morte pode retirar a mente da dizimação que é
a preguiça e a complacência.

Os entraves são ultrapassados temporariamente se se alimentar a corrente
imparável da concentração. Quando os obstáculos já não estão presentes,
a mente entra num estado calmo e estável, caracterizado por uma
consciência sem esforço, um forte sentido de bem"-estar e prontidão para
o trabalho de \emph{vipassanā}. Ao longo do desenvolvimento de
\emph{vipassanā}, quem medita pode, eventualmente, alcançar um ponto
onde as obstruções já não atingem a mente.

Não obstante as várias técnicas disponíveis para lidar com os
impedimentos em sessões de meditações formais, os efeitos
transformadores de meditação só se dão se os praticantes se esforçarem
por aplicar em suas vidas diárias os princípios do Dhamma. Quem leva uma
vida irresponsável ou excessivamente ocupada, depara"-se com o facto de
que a sua forma de vida alimenta os obstáculos em tal medida, que as
técnicas de meditação são impotentes para os ultrapassar. Os elementos
internos e externos de manutenção devem estar em harmonia para que o
progresso ocorra. Por este motivo, os praticantes de meditação devotados
cuidam da qualidade de suas acções e palavras, e simplificam as suas
vidas o mais possível.

\section{Qual o significado de samatha e de vipassanā?}

Samatha (literalmente `lisura') refere"-se a:

\begin{enumerate}
\item
  Práticas de meditação com o fim de ultrapassarem os estados mentais
  menos saudáveis, cultivando sistematicamente os estados mentais de
  sustentação, em particular as qualidades de estar consciente, de estar
  alerta e de empenho.
\item
  A calma vívida e estável da mente que resulta de tais práticas (aqui é
  um sinónimo de \emph{samādhi}).
\end{enumerate}

Vipassanā (literalmente `ver com clareza') refere"-se a:

\begin{enumerate}
\item
  Práticas de meditação que pretendem remover permanentemente estados
  mentais prejudiciais, desenraizando os apegos que lhes dão suporte. No
  vipassanā quem medita investiga as três características da existência
  condicionada: impermanência, dukkha e não"-eu.
\item
  A realização das três características da existência que produzem a
  libertação do apego.
\end{enumerate}

A relação, e a relativa importância entre estes dois tipos de meditação
têm sido fonte de muito debate entre budistas que meditam. É bom dizer
aqui, que uma prática de meditação com sucesso requer um equilíbrio
entre as duas abordagens. Samatha sem vipassanā pode conduzir à
condescendência de estados mentais abençoados; vipassanā sem samatha
pode tornar"-se árida e superficial. Um grande mestre comparou samatha ao
peso da faca, e vipassanā ao seu gume.

\section{Existe alguma técnica de meditação considerada mais eficaz?}

Nenhuma técnica em particular é considerada universalmente eficaz. A
utilidade de cada técnica varia de acordo com os factores físicos e
psicológicos de cada pessoa. Dito isto, o processo de se focar na
respiração tem sido, sempre, a técnica de meditação mais popular na
tradição budista, tendo sido um método que o próprio Buda usou, e que
muito enalteceu. A técnica de meditar com a respiração é imediata e o
seu objecto está sempre à disposição. A forma pela qual a respiração
muda em resposta à atenção que se lhe dá, permite que os praticantes
desenvolvam tanto uma calma interior, quanto uma compreensão da relação
entre o corpo e a mente.

\section{Qual a chave do sucesso para obter uma prática de meditação a longo
  termo?}

O mais importante é não parar. Enquanto os praticantes continuarem a
esforçar"-se por meditar, venha o que vier, estão a acumular condições
para o sucesso. Assim que param, estão a fechar a porta para a paz e
para a sabedoria.

A constância e a regularidade da prática são uma ajuda extremamente
importante. Embora a meditação não seja uma corrida, a mente com passada
de tartaruga terá sempre vantagem sobre a da lebre. Curtos lampejos de
meditação afincada (geralmente em resposta a momentos de crise) seguidos
de longos períodos de negligência, não produzirão resultados duradouros.

\section{Qual a importância de ter um professor?}

As condições ideais para o progresso espiritual são experimentadas por
quem vive numa comunidade liderada por um professor iluminado, mas raros
são, incluindo os monásticos, aqueles a quem é concedida tal
oportunidade. Receber instruções de um professor qualificado, levá"-las
consigo para as pôr em prática, e a seguir encontrar"-se com o professor
a cada momento, para conseguir relatar o progresso, e receber conselhos
e encorajamento, é um procedimento tão concretizável, quanto beneficial.
A competência do professor para assinalar os pontos fracos, as
dificuldades de compreensão e as tendências do estudante para se
desviar, significa que o contacto regular mantido com ele, ou com ela, é
deveras valioso. Períodos ocasionais de retiro com o professor tendem a
ser especialmente profícuos.

Outra abordagem, é a de beneficiar dos imensos ensinamentos sobre
meditação actualmente disponíveis, através dos diversos meios de
comunicação. Pode"-se encontrar informação fiável em livros, DVDs, e na
internet. Na Tailândia, muitos dos programas de Dhamma são divulgados na
rádio e na televisão. Isto pode representar uma grande oportunidade mas,
ao mesmo tempo, pode encorajar superficialmente: algumas pessoas acabam
por acumular uma série de diferentes técnicas, sem se comprometerem com
nenhuma delas em particular. O progresso na meditação requer que se
assuma um método, aplicando"-o consistentemente durante um longo período
de tempo.

\section{Qual o benefício de frequentar retiros de meditação?}

Um retiro de meditação oferece, a quem medita, uma oportunidade de se
dedicar às práticas de meditação ao longo de muitas horas diárias, sob a
supervisão de um professor qualificado, e com o benefício de ter o apoio
de um grupo de pessoas com processos semelhantes. Ao recluir"-se do
ambiente rotineiro, das responsabilidades e dos problemas, num retiro de
sete ou dez dias, pode construir um momento de prática permitindo
vivenciar momentos de calma e de intuição, normalmente inatingíveis nas
suas vidas diárias.

Os retiros têm um efeito rejuvenescedor em praticantes de longo prazo, e
dão autoconfiança aos recém"-chegados. Os noviços nesta prática podem
provar a si próprios que a meditação não se limita a lutar com joelhos
doridos e mentes inquietas, mas que surte efeitos e que eles são
capazes de vivenciar esses resultados.

É difícil, para a maioria das pessoas, manter uma prática de meditação
regular em casa. Ao se frequentar um retiro de meditação, é-lhes dada
uma fundação para se configurarem, e a fé no mérito da meditação, que as
ajudará a suportar períodos de dúvida e de desencorajamento.

\section{É possível meditar ouvindo música?}

A música é, sem dúvida, terapêutica. As emoções que surgem ao ouvir
música, podem aliviar, rápida e eficazmente, as tensões físicas. A
atenção dada à música pode, temporariamente, substituir os estados
mentais desvantajosos. Contudo, quem ouve música não cultiva as
qualidades especiais do esforço, de estar consciente e alerta, que são
particularizadas na meditação budista. É melhor considerá"-la como sendo
um preliminar à meditação mais formal.

\section{É possível meditar e praticar jogging (corrida) ou natação?}

As técnicas de meditação budistas podem ser facilmente aplicadas à
maioria dos exercícios físicos repetitivos. Ter em consciência um
mantra, ou sensações específicas corporais -- por exemplo, as sensações
nas solas dos pés -- pode ser utilmente aplicado enquanto se pratica
corrida ou natação.

\section{A meditação pode ser perigosa?}

Meditar alguns minutos por dia não tem qualquer perigo. Contudo, para
pessoas que sofram de sérios problemas psicológicos, especialmente os
que requerem medicação, não são aconselháveis longos períodos de
meditação sem orientação. Quando as pessoas sofrem de obsessões mentais,
os professores budistas podem desaconselhar a meditação sentada e, em
sua substituição, encorajar a focarem"-se nas práticas de dádiva e de
ajuda aos outros. Neste ponto de suas vidas, a alegria encontrada no
serviço e nas acções de generosidade, juntamente com o auto"-respeito
ganho no cumprimento dos preceitos, pode ter um efeito muito mais
terapêutico do que a aplicação a técnicas de meditação.

\section{Existem alguns objectivos específicos para os budistas leigos?}

O Budismo pode ser entendido como um sistema de educação voluntária. O
nível de compromisso com esta educação é deixado à decisão de cada um.
Muitos budistas leigos ficam contentes com o nível mais básico de
compromisso: o de levarem uma vida boa e moral, cumprindo as
responsabilidades familiares e praticando a generosidade, reforçados com
a fé de que tais irrepreensíveis acções lhes trarão bons resultados
nesta vida, e em vidas futuras.

Mas, não obstante o Buda ter reconhecido a razoabilidade de tal caminho,
considerou que, em última instância, acaba por falhar, não alcançando o
benefício completo proporcionado pela preciosa oportunidade de se ter
nascido humano. Encorajou os budistas leigos a praticarem o nobre
Óctuplo Caminho, na sua totalidade. Embora o desenvolvimento espiritual
seja mais difícil na vida familiar do que num mosteiro, nunca é
impossível, seja qual for o nível a que é assumido, e é verdadeiramente
benéfico. Na verdade, passados dois mil e seiscentos anos, um grande
número de budistas leigos realizou o primeiro nível de iluminação, `a
entrada na corrente'. É a entrada na corrente que se espera ser o
objectivo espiritual apropriado para os budistas leigos sinceros.

Numa ocasião, para assinalar a enorme importância da entrada na
corrente, o Buda afirmou que: se todo o sofrimento vivido em todas as
nossas vidas fosse comparável ao solo no mundo, o sofrimento
remanescente, para quem entrou na corrente, é comparável à terra que se
encontra debaixo de uma unha.

\section{A meditação pode ser praticada na vida diária?}

Tomar um minuto ou dois por dia para restabelecer o estado consciente,
focando"-se na respiração, é um excelente meio de prevenir a acumulação
de stress. Nas situações quotidianas, quem medita precisa ser flexível
na aplicação das técnicas de meditação, pois nenhuma técnica isolada
poderá ser eficaz ou até mesmo apropriada. Quem medita desenvolve uma
série de `meios versáteis' aprendendo a aplicá"-los, a seguir, em
situações diferentes. O princípio fundamental da prática no quotidiano é
dado pelos princípios traçados no caminho pelo factor `Esforço
Correcto'. O praticante procura evitar que surjam estados mentais
perniciosos, libertar"-se dos que já tenham surgido, introduzir na mente
estados mentais benéficos que ainda não tenham surgido e reforçar os
estados benéficos que já tenham surgido.
