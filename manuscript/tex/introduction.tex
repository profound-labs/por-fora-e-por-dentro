\clearpage
\chapter{Introdução}

Para as pessoas que visitam a Tailândia não é fácil que as tradições
budistas, com que se deparam aqui, façam sentido. São poucos os guias
turísticos que sabem explicar os princípios do Budismo com bastante
clareza, e os amigos do Budismo Tailandês têm tendência a ser igualmente
vagos. Este livro pretende oferecer uma introdução aos ensinamentos do
Buda, o que lançará alguma luz num assunto que, para os não"-budistas,
pode parecer tão inesperadamente racional quão exoticamente estranho.

Este não é um livro habitual. Pretende ser tão conciso quanto possível,
e trata num parágrafo assuntos que se encontram tratados noutros livros
em centenas de páginas. É óbvio que se omitiu muita coisa. Aos leitores
interessados em saber mais sobre pontos específicos, é-lhes referido a
lista de recursos que se encontra no fim deste livro.

Ao longo dos últimos 2.600 anos, desenvolveram"-se muitas formas de
Budismo. Este livro trata apenas dos ensinamentos da tradição do Budismo
Theravāda, e em particular da forma do Theravāda da Tailândia (o que
difere em certos detalhes menores da sua expressão de outros países
Theravāda, tais como O Sri Lanka ou Burma). Este livro também foi
escrito sob a perspectiva de um monge particular, que vive dentro da
tradição Theravāda Tailandesa.

Nasci em Inglaterra, mas tenho vivido nos mosteiros da floresta e
ermitérios do nordeste da Tailândia desde 1978. Inevitavelmente, o meu
passado e prática influenciaram as interpretações que aqui se encontram.

\clearpage

Fui bastante afortunado por ter estudado com mestres sábios, e esta
apresentação do Dhamma deve muito a eles, em particular a dois dos
monges mais importantes da era moderna, o Venerável Ajahn Chah e Prha
Brahmagunabhorn (P.A.Payutto). Gostaria de deixar expressa a minha
profunda gratidão a ambos.

\bigskip

{\raggedleft
  Ajahn Jayasāro\\
  Ermitério Janamāra\\
  Março de 2013
\par}
