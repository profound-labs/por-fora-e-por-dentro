\chapterPhotoTwoPageLeft{candles-Gr-crop}

\chapterPhotoTwoPageRight{candles-Gr-crop}

\setlength{\chapterTitleTopSkip}{35mm}
\definecolor{photoChapterText}{gray}{1}

\chapter{Recursos Budistas}

\clearpage
\quoteTitleFmt{O Metta Sutta}

\emph{As palavras do Buddha sobre o Amor e a Compaixão}

\begin{verse}

Eis o que se deve fazer\\
Para cultivar a bondade\\
E seguir a via da paz:\\
Ser capaz e ser honesto,\\
Franco e gentil no falar.\\
Humilde e não arrogante,\\
Contente, facilmente satisfeito,\\
Aliviado de deveres e frugal no seu caminho.

Pacífico e sereno, sábio e inteligente,\\
Sem orgulho, sem exigência por natureza.\\
Que ele nada faça\\
Que os sábios possam vir a reprovar.\\
Desejando: Na alegria e na segurança,\\
Que todos os seres sejam felizes.\\
Quaisquer que sejam os seres vivos,\\
Fracos, fortes, sem excepção\\
Dos maiores aos mais pequenos,\\
Visíveis ou invisíveis,\\
Estejam perto ou estejam longe,\\
Nascidos ou por nascer --\\
Que todos os seres sejam felizes!

Que ninguém engane ninguém,\\
Ou despreze alguém em que estado fôr.\\
Que ninguém por raiva ou má-fé,\\
Deseje mal a alguém.\\
Assim como uma Mãe protege o filho,\\
Com sua vida, seu único filho,\\
Igualmente de coração infinito,\\
Se deveria estimar todo o ser vivo;\\
Irradiando ternura por todo o mundo:\\
Acima ao mais alto céu,\\
E abaixo às profundezas;\\
Irradiante e sem limites,\\
Livre de ódio e má-fé.\\
Seja parado ou a andar,\\
Sentado ou deitado,\\
Livre de torpor,\\
Esta é uma lembrança a manter.

Diz-se esta ser a sublime permanência.\\
O puro de coração, com clareza de visão,\\
Ao não insistir em ideias fixas,\\
Liberto dos desejos dos sentidos,\\
Não voltará a nascer neste mundo.

{\raggedleft
Khuddaka Nikāya, Sutta Nipāta 9
\par}

\end{verse}

\section{Pode recomendar alguns websites onde se aprenda o que é o Budismo?}

http://www.buddhanet.net

http://www.accesstoinsight.org

http://www.forestsangha.org

http://www.buddhistteachings.org

http://www.suanmokkh.org

http://blogs.dickinson.edu/buddhistethics/

http://online.sfsu.edu/rone/Buddhism/Buddhism.htm

http://www.buddhistchannel.tv

https://suttacentral.net

https://en.bia.or.th

\section{Pode recomendar alguns livros sobre o Budismo?}

Há imensos ebooks excelentes, disponíveis para descarregar sem custos.
Os primeiros cinco websites têm uma grande selecção.

Muitos dos livros na lista seguinte também estão disponíveis sob formato
de ebook, e através dos vendedores de livros online.

\section{Discursos do Buda:}

\emph{The Middle Length Discourses of the Buddha}\\
Bhikkhu Bodhi e Bhikkhu Nyanamoli

\emph{The Connected Discourses of the Buddha}\\
Bhikkhu Bodhi

\emph{The Numerical Discourses of the Buddha}\\
Bhikkhu Bodhi

\emph{The Long Discourses of the Buddha}\\
Maurice Walshe

\section{De uma perspectiva mais geral:}

Todos os livros do Bhikkhu P.A. Payutto.

\emph{The Life of the Buddha}\\
Bhikkhu Nyanamoli

\emph{Great Disciples of the Buddha}\\
Helmut Hecker

\emph{The Foundations of Buddhism}\\
Rupert Gethin

\emph{Buddhist Religions}\\
Richard H. Robinson, Willard L. Johnson, Bhikkhu Thanissaro

\section{Sobre Meditação}

Os trabalhos de grandes mestres tailandeses, a citar: Ajahn Maha Bua,
Ajahn Chah, Ajahn Buddhadasa, estão disponíveis para download grátis,
bem como os dos monges ocidentais da tradição tailandesa, tais como
Ajahn Sumedho e Thanissaro Bhikkhu.

https://jayasaro.panyaprateep.org

https://www.facebook.com/jayasaro.panyaprateep.org

https://www.youtube.com/ThawsiSchool

Podcasts palavra-passe: Jayasaro

iBook Store: Panyaprateep Foundation, ou palavra-passe: Jayasaro
