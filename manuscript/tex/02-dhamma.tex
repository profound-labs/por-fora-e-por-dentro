\chapterPhotoTwoPageLeft{dhamma-hills-Gr-crop}

\chapterPhotoTwoPageRight{dhamma-hills-Gr-crop}

\setlength{\chapterTitleTopSkip}{10mm}
\definecolor{photoChapterText}{gray}{0}

\chapterNote{%
  O Dhamma foi bem apresentado pelo Abençoado,\\
  Aparente aqui e agora,\\
  Intemporal,\\
  Encorajando à investigação\\
  Conduzindo para a frente,\\
  A ser experimentado pelo sábio.
}

\chapter{O Dhamma}

\section{O que significa `Dhamma'?}

O Dhamma\footnote{`Dhamma' em Pali é o mesmo que `Dharma' em Sânscrito.}
refere"-se a:

\begin{enumerate}
\item
  A verdade das coisas, `como as coisas são', a verdadeira realidade.
\item
  Os ensinamentos do Buda que iluminam essa verdade, e que pormenorizam
  o caminho conducente à sua experiência directa.
\end{enumerate}

\section{O que são as Quatro Nobres Verdades?}

Todos os ensinamentos do Buda estão englobados no que se chama as Quatro
Nobres Verdades, segundo ele explicou, tal como a pegada de qualquer
animal da floresta cabe dentro da pegada de um elefante. Estas verdades
revelam o problema fundamental da nossa existência, bem como a sua
resolução.

\begin{enumerate}[topsep=0pt]
\item Existe \emph{dukkha}
\end{enumerate}

Dukkha é geralmente traduzido como `sofrimento', mas na verdade tem um
significado bem mais profundo do que o que está implicado nessa palavra.
Dukkha refere"-se à insatisfação crónica da existência não iluminada.
Cobre todo o espectro da experiência, desde severas dores físicas e
emocionais até ao mais subtil sentido de desconforto e de carência.

\begin{enumerate}[resume,topsep=0pt]
\item Há uma causa para dukkha
\end{enumerate}

Dukkha não é uma dificuldade humana inalterável. Depende de certas
causas e condições, em particular dos desejos que surgem de uma
percepção fundamentalmente errada da nossa natureza humana.

\begin{enumerate}[resume,topsep=0pt]
\item Existe a extinção de dukkha
\end{enumerate}

Existe um fim total de dukkha, um estado de libertação e de verdadeira
felicidade.

\begin{enumerate}[resume,topsep=0pt]
\item Existe o caminho conducente à cessação de dukkha
\end{enumerate}

Através do cultivo do Óctuplo Caminho, dukkha é compreendido, as suas
causas são abandonadas e ocorre o seu término. Esta via envolve uma
educação ou uma prática em todos os aspectos da nossa vida, tanto no
interior como no exterior. Os oito factores são os que seguem:

\begin{enumerate}
\item Entendimento Correcto
\item Pensamento Correcto
\item Discurso Correcto
\item Acção Correcta
\item Meio de Subsistência Correcto
\item Esforço Correcto
\item Consciência Correcta
\item Concentração Correcta
\end{enumerate}

\section{Por favor explique mais detalhadamente o que é o Óctuplo Caminho.}

O Óctuplo Caminho é a educação holística, ou o treino do corpo, da fala
e da mente, que culmina no despertar.

O Entendimento Correcto refere"-se às crenças, visões, ideias, valores
que estão em harmonia com a forma como as coisas são. Inicialmente os
seus elementos mais importantes são a confiança na 1) capacidade humana
para se iluminar e 2) lei do kamma.\footnote{`Kamma' em Pali é o mesmo que `karma' em Sânscrito.}

O Pensamento Correcto refere"-se aos pensamentos consistentes com a Visão
Correcta. Caracterizam"-se por uma ausência de pensamentos indesejáveis,
em particular os do tipo 1) sensual, 2) hostil, ou 3) cruel. O
Pensamento Correcto inclui a aspiração de se libertar de todas as
aflições interiores, e alcançar pensamentos de bondade e compaixão.

O Discurso Correcto é um discurso verdadeiro, útil e intemporal, cortês
e gentil na sua intenção. É um discurso livre de 1) mentira, 2) rudeza,
3) calúnia e 4) maledicência.

A Acção Correcta refere"-se a acções que não prejudicam o próprio nem os
outros. Basicamente tem que ver com abster"-se de 1) matar, 2) roubar e
3) ter má conduta sexual.

O Meio de Subsistência Correcto significa sustentar"-se de forma a não se
prejudicar a si próprio, nem aos outros. Más formas de ganhar a vida,
enunciadas nos textos, incluem vender: 1) armas, 2) seres humanos, 3)
carne e peixe, 4) drogas e 5) venenos.

O Esforço Correcto refere"-se a tentar:

\begin{enumerate}[topsep=0pt,parsep=0pt]
\item Evitar que cheguem à mente pensamentos e emoções inadequados.
\item Reduzir e erradicar pensamentos e emoções inadequados que já tenham chegado à mente.
\item Introduzir antecipadamente na mente pensamentos e emoções adequados.
\item Manter e desenvolver pensamentos e emoções adequadas, já presentes na mente.
\end{enumerate}

A Consciência Correcta refere"-se a manter uma consciência alerta, serena
e confiante, no momento presente, em particular:

\begin{enumerate}[topsep=0pt,parsep=0pt]
\item No corpo humano
\item Sob o aspecto em que afecta a experiência: agradável, desagradável ou neutro.
\item No estado mental
\item Nos fenómenos mentais, em como se relacionam com o caminho para o despertar de Buda.
\end{enumerate}

A Concentração Correcta refere"-se à estabilidade interior, à clareza e à
paz experimentada nos quatro estádios da `absorção meditativa', ou `jhāna'.

O primeiro jhāna é caracterizado pelos cinco `factores jhāna': uma
sustentada atenção inicial, manutenção da atenção no objecto de
meditação, entusiasmo, felicidade, e focagem da mente. Consoante a mente
se torna mais refinada, os factores mais densos de jhāna desvanecem"-se.
O segundo jhāna é alcançado com o esvanecer da atenção inicial. Ao
desaparecer o entusiasmo, assinala"-se o atingir do terceiro jhāna. Com a
perda da felicidade, a mente entra no quarto e mais subtil estado de
jhāna, que se distingue pela inabalável equanimidade.

\section{O que significa `tomar refúgio'?}

A vida é cheia de dificuldades, nunca está liberta de sofrimento, ou
pelo menos da possibilidade de este surgir. Ao sentirem"-se inseguros, e
num estado crónico de carência, os seres humanos desejam ardentemente
segurança. Alguns procuram"-na adoptando um sistema de crença ou o
conforto de rituais. Igualmente popular é o caminho das distracções:
perseguindo prazeres sensoriais, riqueza, fama, poder e estatuto. Sob o
ponto de vista budista nenhuma destas estratégias atinge esse alvo. Nem
os prazeres dos sentidos, nem o sucesso mundano poderão satisfazer as
necessidades humanas mais profundas. A fé nos dogmas ou a realização de
rituais não conseguem providenciar um refúgio verdadeiro. Enquanto as
pessoas não obtiverem clareza na compreensão de suas vidas e continuarem
a agir com pouca sensatez, nunca se sentirão em segurança.

Tomar refúgio na `Jóia Tripla' (Buda, Dhamma, Sangha) considera"-se
ser o primeiro passo para alcançar a libertação do sofrimento, bem como
das suas causas, pois fornece, aos esforços feitos pelos budistas, uma
base de suporte e uma direcção para alcançarem esse objectivo. O acto de
se refugiarem assinala o primeiro passo de compromisso no caminho do
Buda. Os budistas afirmam refugiar"-se no Buda, como seu professor e
guia, no Dhamma, os ensinamentos, como seu caminho, e no Sangha, os
discípulos iluminados, como sendo a inspiração para o caminho.

\section{Porque é que os ensinamentos budistas são referidos frequentemente como
  o Caminho do Meio?}

O `Caminho do Meio' é um termo usado pelo Buda em dois contextos
distintos. Primeiro, como característica"-cerne de seu ensinamento --
todas as coisas surgem e desaparecem devido às causas e condições --
como um caminho do meio entre os extremos do aniquilacionismo (a crença
de que tudo termina com a morte) e a do eternalismo (a crença que a
morte é seguida de felicidade ou de condenação eternas).

Segundo, o Buda apresentou o Óctuplo Caminho como um caminho médio entre
os extremos da indulgência sensorial e do vazio asceticismo, (`sem dor
não há benefício'). Contudo, seria um erro olhar para isto como sendo
apenas um ensinamento de moderação. Pelo contrário, o Caminho do Meio
deve ser compreendido dentro do conceito do esforço geral que leva ao
abandono dos estados mentais inadequados, ao cultivo dos estados mentais
adequados, e à libertação da ignorância e da ilusão. O Caminho do Meio
não se encontra ao se buscar um ponto médio entre os dois extremos, mas
antes, encontra"-se sempre presente naquilo que qualquer prática
espiritual possibilita como uma progressão excelente para o despertar.

\section{O que é que o Budismo ensina sobre a natureza da felicidade?}

Os seres humanos podem obter dois tipos de felicidade: a que depende dos
estímulos externos, a que não depende disso. O primeiro tipo de
felicidade é vivido, ao seu nível mais básico, nos prazeres sensoriais:
vendo, ouvindo, cheirando, saboreando e tocando coisas agradáveis.
Também inclui as emoções positivas que vivemos através das relações
pessoais, realizações mundanas e do estatuto social.

O segundo tipo de felicidade é conhecido com o desenvolvimento
espiritual. Inicialmente é desfrutado pelo cultivo da generosidade e da
disciplina moral, mas atinge os seus níveis mais profundos com a
meditação. Meditadores experientes reconhecem o entusiasmo e a
felicidade que acontecem numa mente focada, como sendo
inquestionavelmente superiores aos prazeres que dependem dos sentidos
mais densos. Mas os estados meditativos mais refinados não representam a
felicidade suprema. É no gradual abandono dos estados mentais nocivos --
a causa original do sofrimento -- que o praticante descobre um sentido
estável e sublime de bem"-estar. Isto é considerado como sendo um estado
superior de felicidade, experienciado como uma expressão natural de uma
mente refreada, mais do que como uma experiência de passagem sujeita a
ganhos ou perdas.

Os budistas leigos são encorajados a procurar, moderadamente, a
felicidade no mundo, compatível com o acesso à felicidade interior; e a
renunciar à complacência dos prazeres mundanos, que distraem a mente do
trabalho espiritual.

\section{Porque é que tão poucas pessoas parecem ser verdadeiramente felizes?}

O Buda ensinou que todos os seres vivos nascem com um desejo instintivo
de evitar o sofrimento e de vivenciar a felicidade. O problema é que,
por ausência da sabedoria, continuamos a agir criando condições para
sofrer, bem como negligenciando as acções que criariam as condições para
sermos felizes. Procuramos a felicidade nas coisas que inevitavelmente
nos vêm a desapontar; evitamos o que nos conduziria a um bem"-estar
duradouro. Em resumo, somos o nosso pior inimigo.

São poucas as pessoas que consideraram seriamente a natureza da
felicidade. Daqueles que o fizeram, ainda menos são as que se
comprometeram a erradicar sistematicamente os seus obstáculos internos e
a cultivar as condições que a suportam. Não é de surpreender que tão
poucas pessoas sejam verdadeiramente felizes.

Uma das premissas básicas do Budismo é a de que quanto mais claramente
virmos a natureza das coisas, menos sofreremos, e mais felizes seremos.
Na verdade, o Buda referiu"-se ao \emph{Nibbāna}, o objectivo da prática
budista, como a `suprema felicidade'. A felicidade mundana é fugaz e
nada fiável. A felicidade de uma mente controlada é um refúgio
duradouro.

\section{Os budistas falam muito do momento presente. Isso não entra em conflito
  com o que se aprende nas experiências passadas, ou com planear o futuro?}

O passado e o futuro encontram"-se no momento presente: o passado como
memória, o futuro como pensamento e imaginação. Qualquer acesso a
experiências passadas, qualquer decisão relativa ao futuro é actividade
mental que ocorre inevitavelmente no momento presente -- não há escolha,
é tudo quanto temos. O problema é que ao não se estar consciente da
memória, como memória, e do pensamento, como pensamento, facilmente
perdemo"-nos neles. Quando perdemos a presença de espírito desta
maneira, a nossa vida torna"-se uma abominável sombra de si própria.

Quanto mais conscientes estivermos do momento presente, menos confusa se
torna a mente, e mais fácil se torna aprender com as experiências
passadas, e planear o futuro com sabedoria.

\section{O que é o mérito?}

O mérito (\emph{puñña}) significa a purificação interna que ocorre por
acções virtuosas do corpo, das palavras e da mente. As acções meritórias
elevam e enobrecem a mente, e são acompanhadas por um sentimento de
bem"-estar.

Na Tailândia, a expressão popular de `criar mérito' (\emph{tham boon)}
refere"-se geralmente à contribuição de ofertas para a ordem monástica.
Tais ofertas, quando dadas com a motivação certa, podem efectivamente
ser meritórias, mas o mérito não se restringe apenas a esses actos.

As acções generosas são valiosas porque reduzem o poder do apego
egoísta, e ensinam a alegria da dádiva. Praticar os preceitos é
meritório porque enfraquece o impulso de nos magoarmos e de magoar os
outros, porque habitua a mente a libertar"-se de remorsos e a sentir
respeito por si próprio. Mas o tipo de mérito mais importante vem da
prática do Óctuplo Caminho, particularmente da prática da meditação.
Meditar com regularidade implica comprometer"-se no cultivo das
competências da vida. Significa responsabilizar"-se directamente pelo
abandono das causas do sofrimento, e por exponenciar a paz, a sabedoria
e a compaixão. Como a meditação é que efectua a maior transformação da
mente, ela é o mais portentoso gerador de mérito.

O Buda ensinou que os frutos do mérito não terminam na morte, mas
contribuem para um bom renascer. Embora o Buda enfatize a importância da
libertação do ciclo do nascimento e da morte (saṃsāra), também reconhece
que, para aqueles que não estão preparados para tal caminho, a
acumulação de mérito como benefício desta e doutras vidas futuras é uma
via compreensível (e nada pouco inteligente) a seguir.

\section{Por vezes afirma"-se que o Budismo é uma ciência. O~que é que isso
  significa?}

Existem semelhanças entre as práticas contemplativas budistas e o método
científico, na rejeição da fé cega e na ênfase da investigação imparcial
dos fenómenos, essencial a ambos métodos de investigação. Contudo,
também há diferenças. Nas suas investigações, a ciência limita"-se a
estudar tudo quanto é publicamente verificável, possível de medir, e que
pode ser repetido sempre que se deseja. A investigação introspectiva
levada a cabo pelos praticantes de meditação budistas não o é.
Actualmente, a maioria dos cientistas assumem, como premissas básicas do
seu trabalho, um número de asserções não comprovadas, com as quais os
budistas não concordam. A mais notável destas é a crença que a mente é
um fenómeno meramente criado pelos trabalhos do cérebro.

As hipóteses e as teorias surgem na mente humana - não são embebidas no
mundo externo. A experiência subjectiva é a característica principal da
nossa vida. A contenda budista defende: qualquer busca das verdades
perpétuas que ignore este facto, ficará para sempre circunscrita ao sucesso
parcial.

Apesar destas e doutras diferenças entre o Budismo e a ciência, há que
reconhecer que muitos budistas sentem que os seus pontos de vista estão
em maior consonância com os dos cientistas seculares, do que com a visão
da maioria das outras tradições religiosas.

\section{Será o Budismo uma religião pessimista?}

Pessimismo, numa acepção mais comum, significa `uma tendência para ver
o aspecto pior das coisas, ou acreditar que vai acontecer o pior; uma
falta de esperança ou de confiança no futuro', e sob uma perspectiva
filosófica: `uma crença de que este mundo é tão mau como deveria de
ser, ou de que o mal, em última instância, prevalecerá sobre o bem'.

Nenhum destes significados se aplica aos ensinamentos budistas. O Buda
ensinou que tudo quanto surge desaparece de acordo com as causas e as
condições. Se, numa determinada situação, prevalecerem as causas e as
condições para que aconteça o pior, então o pior acontecerá; se
prevalecem as causas e as condições para que aconteça o melhor resultado
possível, então surgirá o melhor desfecho. Ele enfatizou que se deve
aprender a ver tudo com clareza, em vez de se adoptarem atitudes
unilaterais.

O Buda, ao compreender a natureza causal dos fenómenos, não postulou
valores absolutos de bem e de mal, opondo"-se entre si numa guerra sem
fim. Por isso, há que descartar a ideia de ele ter ensinado o triunfo
final de um dos lados da luta, coisa que, primeiro de tudo, ele não
reconheceu existir. Os budistas defendem que, se uma chávena de chá
tiver um sabor salgado, mesmo que seja o mais desagradável possível, não
é uma evidência de um universo essencialmente maligno. É simplesmente o
resultado de alguém que se enganou no recipiente, pegando no do sal, em
vez do açúcar.

\section{Mas afinal o Budismo não trata só do sofrimento?}

O Buda disse que, todos os seus ensinamentos, tradicionalmente contados
como 84.000, se poderiam reduzir a apenas dois: sofrimento e o fim do
sofrimento. O sofrimento, no sentido de aflição física ou mental, é só a
expressão mais grosseira de dukkha. A relação que existe entre a palavra
`sofrimento', em Inglês (e em Português), e o conceito de dukkha,
em Pāli, pode ser vista como a comparação entre vermelho vivo e cor.
Dukkha também pode ser traduzido como um sentido crónico de ausência, ou
como um defeito, ou incompletude de experiência. Neste sentido, dukkha é
experiência vista como `não"-Nibbāna'. Por este motivo, até mesmo os
estados mentais mais sublimes ainda são considerados como existindo no
domínio de dukkha, porque como são fenómenos condicionados, o apego que
se lhes tem não deixa que aconteça a derradeira paz.

Posto de uma forma mais simples, dukkha pode ser expresso como `uma
ausência de verdadeira felicidade'.

O Buda ensinou a via para a cessação do sofrimento, mas enfatizou que a
libertação do sofrimento só seria possível, se ele fosse confrontado e
completamente compreendido na sua natureza. Na Primeira Nobre Verdade, o
Buda afirma que a vida do ser comum não iluminado se caracteriza por
dukkha, devido aos anseios que acompanham o desconhecimento de como as
coisas são.

\section{É correcto dizer que o Budismo nos ensina a renunciar a todos os
  desejos?}

O Budismo distingue dois tipos de desejos: o primeiro (tanhā), a ser
abandonado, e o segundo (chanda), a ser cultivado.

Tanhā é o desejo que surge de um mal"-entendido básico sobre como são as
coisas: o facto de se ver permanência, felicidade e individualidade,
onde não existem. O desejo pelos prazeres, a obter através de posse, de
descartar algo, e de se tornar em algo, é tanhā. Tanhā leva ao
sofrimento pessoal e é a base de quase todos os males sociais.

Chanda é o desejo que surge de uma compreensão correcta de como as
coisas são. No seu cerne reside a aspiração à verdade e ao bem. O desejo
de fazer bem, de actuar bem, de agir com bondade, de agir com sabedoria
-- todos os desejos baseados na aspiração à verdade e ao bem conduzem à
realização pessoal e a sãs comunidades.

A distinção entre chanda e tanhā não é filosófica, mas psicológica. Ao
se observar de perto a crua experiência de vida, a distinção entre
desejos que levam à felicidade genuína, e os que não levam, torna"-se
cada vez mais clara.

\section{O que significa `largar'?}

O Buda ensinou"-nos a observar como nós criamos sofrimento constantemente
para nós próprios, apegando"-nos ao corpo e seus sentidos, aos
sentimentos, percepções, pensamentos, emoções, como sendo `eu' ou `meu'.
Aprender a abandonar esse hábito, é aprender a `largar'. Tal não é
possível por um acto de vontade. O acto de largar ocorre naturalmente
quando a mente treinada se torna suficientemente acutilante para se
aperceber que não existe nada na experiência directa que possa
corresponder ao conceito de `eu' e `meu'.

`Eu' e `meu' não são, contudo, meras ilusões; são convenções sociais
extremamente úteis, e o Buda ensinou a respeitá"-las como tal. Embora o
corpo, por exemplo, estritamente falando seja `não meu', não quer dizer
que deva ser negligenciado. Largar o corpo não significa que se deva
deixar de praticar exercício, de tomar banho ou de ter uma dieta
saudável. Significa não permitir que a vida seja definida em termos
corporais. Significa libertar"-se de toda a ansiedade, insegurança e
vaidade, de todo o medo de envelhecer, adoecer e morrer, que acompanham
uma relação nada sábia com o corpo.

`Largar' é também um termo usado para um esforço inteligente. Ao
sabermos que nenhum esforço que façamos existe em vão, que será sempre
afectado de alguma forma pelas condições sobre as quais não temos
controlo, largamos as nossas exigências e expectativas relativas ao
futuro. Criamos as melhores condições possíveis para atingir os nossos
objectivos, e depois libertamo"-nos dos resultados.

\section{Como é que alguém se torna budista?}

Falando de forma prática, alguém torna"-se budista quando, ao se refugiar
no Buda, no Dhamma, e no Sangha, começa a aplicar"-se no estudo e a
aplicar os ensinamentos de Buda na sua vida.

Nos países budistas, como a Tailândia, nunca houve cerimónias
específicas, onde as pessoas possam afirmar formalmente a sua devoção ao
Budismo. De certa forma, talvez isto se deva por o Budismo não ser uma
religião baseada na adopção de determinadas crenças; e, em parte, também
por não existir qualquer proselitismo budista, poucos são os
recém"-convertidos. Para o melhor e para o pior, as pessoas encaram a sua
identidade budista como algo garantido, assumindo que são budistas desde
o dia em que nasceram.

A situação é algo diferente na Índia. Aproximadamente nos últimos
setenta anos, um grande número de Dalit (antes chamados de `intocáveis')
converteram"-se ao Budismo seguindo o exemplo do seu líder Dr. Ambedhkar.
Levaram"-se a cabo imensas cerimónias de conversão, onde se formalizou a
aceitação dos refúgios em Buda, Dhamma e Sangha, e um compromisso de
viver de acordo com os cinco preceitos. (Esta fórmula de requerer os
refúgios e os preceitos a partir da comunidade monástica é incorporada
em quase todas as cerimónias budistas na Tailândia.)

\section[O que são conta\kern-0.1pt minações?]{O que são contaminações?}

A mente destreinada é presa de vários estados mentais que arruínam o seu
esplendor natural. Estes incluem as diversas formas de cobiça, inveja,
raiva, ódio e animosidade, entorpecimento e agitação, complacência,
confusão, arrogância, desprezo e preconceito, e apego cego a pontos de
vista e crenças. Felizmente, nenhum destes estados mentais está
irrevogavelmente conectado à mente; qualquer um pode ser eliminado pela
prática do Óctuplo Caminho. Estes estados mentais negativos e
prejudiciais são chamados `\emph{kilesa}', na língua Pāli, geralmente
traduzidos como `contaminações ou corrupções'.

Neste livro a expressão `aflições mentais' tem sido preferida a
`contaminações'. As razões desta interpretação não ortodoxa deve"-se, em
primeiro lugar, ao facto de as contaminações serem geralmente entendidas
como sendo irreversíveis, mas as kilesa, não o são; em segundo lugar,
porque `aflições mentais' é um termo actual e poderoso que ilumina as
atitudes descuidadas tendo em vista kilesa; em terceiro lugar, porque
admite a gradação: podemos falar de algo com sendo levemente aflitivo, e
de algo muito aflitivo.

\section{O Budismo é uma religião ou uma filosofia?}

O Budismo pode ser desconcertante para algumas pessoas educadas dentro
de algumas grandes tradições monoteístas, tais como o Cristianismo ou o
Islamismo. Embora as tradições budistas tenham espaço para a devoção e a
cerimónia, o Budismo não tem dogmas, nem uma escritura única. Não
envolve adoração a um deus. O que o Budismo tem é um conjunto de
ensinamentos, que noutras tradições seria considerado dentro do domínio
da filosofia e da psicologia. Por este motivo têm existido muitas
dúvidas se o Budismo é uma religião, ou não.

O Budismo não se configura, de modo algum, no molde de religião criado
no mundo ocidental; se tal significa que o Budismo não é de modo algum
uma religião, ou se é simplesmente um tipo diferente de religião, é um
assunto em debate. Ao colocar o argumento na segunda possibilidade,
pode"-se dizer que, não obstante as religiões que se desenvolveram no
Médio Oriente sejam essencialmente sistemas de crenças, o Budismo é um
sistema educativo.

\section{Existem algumas escrituras budistas?}

O Tipițaka (literalmente, `os três cestos') é uma colecção de textos
fundamentais do Budismo Theravāda preservados na antiga língua da Índia,
o Pāli. Na tradução inglesa os Tipițaka cobrem à volta de 20.000 páginas
impressas. Os Tipițaka dividem"-se em três secções:

\textbf{O Vinaya Piṭaka}\\
A colecção de textos contendo o código disciplinar para monges e monjas,
e as instruções para gerir os assuntos monásticos. As últimas incluem,
por exemplo, as secções de etiqueta monástica, cerimónias e a relação
correcta a ter com os `quatro bens essenciais': a indumentária, a
comida da mendicância , o local de residência e os remédios, bem como os
procedimentos para a ordenação de novos membros, e a resolução de
disputas.

\textbf{O Sutta Piṭaka}\\
A colecção de suttas, ou discursos. Inclui todos os ensinamentos
registados que o Buda ensinou de Dhamma, juntamente com um pequeno
número de discursos dados pelos seus discípulos. O Sutta Pițaka está
dividido em cinco Nikāyas, ou colecções:

\begin{packeditemize}
\item Dīgha Nikāya -- a `colecção dos discursos longos'
\item Majjhima Nikāya -- a `colecção dos discursos de duração média'
\item Saṃyutta Nikāya -- a `colecção temática'
\item Aṅguttara Nikāya -- a `colecção numerada'
\item Khuddaka Nikāya -- a `miscelânea'
\end{packeditemize}

\textbf{O Abhidhamma Piṭaka}\\
Uma revisão e sistematização dos princípios axiais apresentados no Sutta
Pițaka.

\section{Qual é a essência do Budismo?}

O Buda respondeu a esta pergunta com uma poderosa analogia. Disse que
fosse qual fosse o mar, o oceano, de onde se tirasse uma amostra de
água, ela teria sempre o mesmo sabor salgado; assim, qualquer que seja o
ensinamento de Buda, ele revela o sabor único da libertação. A
libertação, ver"-se livre de dukkha e de suas causas, esta é a essência
do Budismo.

\section{Qual é o objectivo final da prática do Dhamma?}

Os resultados da prática do Dhamma podem ser expressos, tanto pela
negativa, como pela positiva. Num sentido negativo, o resultado é a
libertação de todo o sofrimento e de todas as causas do sofrimento,
nomeadamente dos estados mentais nocivos enraizados na cobiça, no ódio e
na ilusão. Num sentido positivo, é a perfeição da sabedoria, da
compaixão e da liberdade interior.

\section{Os budistas acreditam em Deus?}

Como a definição de Deus varia ao longo das diversas tradições
religiosas, não existe uma resposta imediata para esta pergunta. Embora
a noção de uma personificação de um deus criador seja claramente
incompatível com os ensinamentos budistas, alguns dos conceitos mais
abstractos de `Deus' podem"-se reconciliar com eles de alguma maneira.

\section{Qual a credibilidade da autenticidade dos textos budistas, dado terem
  sido transmitidos oralmente durante os primeiros séculos após a morte de
  Buda?}

A transmissão oral dos textos budistas pode ter resultado mais numa
força, do que numa fraqueza. Quando os textos são preservados por grupos
monásticos, cantando"-os em conjunto a intervalos regulares, a
probabilidade de erros de omissão ou de emendas deliberadas é
minimizada. Embora haja que reconhecer que não existem evidências
inabaláveis para a autenticidade dos textos antigos, também há, todavia,
um número de boas razões para neles confiar.

Em primeiro lugar, como foi explicado antes, existe uma coerência
interna e uma ausência de contradição nos discursos de Buda, que são
notáveis tendo em conta o imenso material, centenas de vezes superior,
por exemplo, ao Novo Testamento Cristão. As mesmas colecções de
ensinamentos preservadas por diversas escolas budistas, em diferentes
línguas, demonstram um elevado grau de correspondência.

Os ensinamentos de Buda não se destacam, nem entram em decadência, por
causa de acontecimentos históricos particulares. Descrevem um sistema
educativo para o corpo, a fala e a mente, conducente ao despertar. Ao
longo dos últimos séculos muitos homens e mulheres, monásticos e leigos,
puseram estes textos em prática, e provaram a si próprios a sua verdade
e eficácia. Em último lugar, é por esta razão que os budistas confiam na
autenticidade dos ensinamentos de Buda, que têm vindo a ser transmitidos
até aos dias de hoje.

\section{O Budismo foca"-se demasiado no indivíduo, e falha na dimensão social?}

O termo `Budismo' é de uso recente. O próprio Buda referia"-se aos seus
ensinamentos como Dhamma"-Vinaya, usando `Vinaya' para se referir aos
meios pelos quais o ambiente externo pode ser ordenado de forma a criar
óptimas condições para o estudo, a prática e a realização do Dhamma. O
Vinaya atinge o seu apogeu nas regras e regulamentos que governam a vida
dos monásticos budistas, mas também se aplica à sociedade em geral.
Nesta última acepção, o Vinaya enforma textos, costumes e convenções que
defendem a redução da avidez, do ódio, e da ilusão nas comunidades, e
encorajam o desenvolvimento da justiça, da paz e da harmonia.

Os estudantes dos textos budistas fundamentais deparam"-se com um grande
número de ensinamentos que lidam com a dimensão social do Dhamma. Esta
área do Budismo tem sido provavelmente negligenciada pelos escritores
ocidentais, que se têm interessado mais pelos ensinamentos de meditação.
Ao desejarem um Budismo livre da `bagagem cultural' asiática, acabam,
por vezes, numa visão incompleta e reducionista do Dhamma"-Vinaya.

Para sermos justos, temos de admitir que os chefes das nações do Budismo
moderno têm caído no mesmo erro. Na Tailândia, os deuses do mercado
livre têm vindo a exercer muito mais influência, do que os princípios do
Vinaya. Os ganhos a curto prazo são geralmente vistos como sendo
objectivos mais práticos e recompensadores, do que os de bem"-estar a
longo prazo.

\section{Quanto tempo é preciso para alcançar a iluminação?}

Esta questão tem de ser respondida usando uma velha história:

Um monge viaja pelo interior. Pergunta a uma velhinha sentada na berma
da estrada, quanto falta para chegar à montanha. Ela ignora"-o. Ele
pergunta de novo, e novamente é ignorado por ela. E o mesmo acontece
pela terceira vez. O monge assume que a mulher deve ser surda. Ao
recomeçar a andar, ouve ela a gritar"-lhe: `Sete dias!' Ele
retorque"-lhe: `Avozinha, eu tinha"-lhe feito essa pergunta por três
vezes, e ignorou"-me. Porque é que esperou que eu voltasse a caminhar
para me gritar a resposta?' A idosa senhora disse: `Antes de lhe
poder responder, tinha de ver qual a velocidade do seu andar, e a
determinação que aparentava ter.'

Os budistas que estão convictos que existe algo chamado iluminação, que
têm potencial para o alcançar e que seguem a via para essa realização,
dedicam pouco tempo a especular sobre o tempo que é necessário para o
alcançar. Sete dias, sete meses, sete anos, sete vidas -- qualquer que
seja o tempo que leva não há outro caminho.

\section{Resumidamente o que é a lei do kamma?}

O Buda disse que a essência do Kamma é a intenção. A lei do kamma
exprime a dimensão moral da lei da causa e efeito. Qualquer acção
intencional realizada pelo corpo, fala, ou mente produz resultados de
acordo com essa intenção. Posto de forma mais simples: as boas acções
têm bons resultados; as más acções têm maus resultados. As acções
provocadas por estados mentais nocivos enraizadas na ganância, ódio e
ilusão contribuem para um sofrimento futuro. As acções que provêm da
sabedoria e da compaixão contribuem para a felicidade futura.

\section{Tudo quanto acontece na vida está destinado a acontecer, ou será que
  existe o chamado livre arbítrio?}

O Buda rejeitou a crença de que tudo quanto existe na vida está
predestinado, pré ordenado por um poder sobrenatural. Também encorajou
os seus discípulos a verem como uma ideia baseada no exercício de livre
arbítrio desaparece perante uma análise profunda da mente e do corpo.

Consoante os momentos da vida, a experiência ganha um tom diferente:
agradável, desagradável ou neutro. Ao não usarmos a concentração e a
sabedoria, reagimos ao agradável com apego, ao desagradável com
rejeição, e ao neutro com negligência. Desta forma, a nossa vida é
largamente determinada pelas reacções habituais ao material em bruto que
experimentamos. Com concentração e sabedoria, reconhecemos o tom
efectivo da experiência tal como é, mas tomamos decisões baseadas num
critério mais inteligente. Desta forma, poder"-se-á conhecer uma certa
libertação do que é oferecido.

\section{Por favor exemplifique os trabalhos da lei do kamma.}

Todos os dias realizamos tantos actos provenientes da vontade, a nossa
vida é um fluir de volições tão complexo, que o efeito de qualquer acto
específico é raramente óbvio. Usando uma analogia, se um balde de ácido
estivesse para ser lançado num rio, saberíamos de certeza que o nível do
pH da água seria reduzido a um certo nível. Mas o grau de mudança
observável dependeria de outras substâncias que tivessem sido
introduzidas na água. Se a água já fosse muito ácida, ou muito alcalina,
o efeito poderia não ser tão óbvio.

Embora os efeitos externos das acções kámicas individuais possam não ser
facilmente verificados, a um nível interno já é uma história diferente.
Podemos facilmente observar que, sempre que nos zangamos, aumentamos a
probabilidade de podermo"-nos consentir tal, da mesma forma, no futuro.
Criamos e alimentamos hábitos e traços de personalidade através de um
pingar constante de acções volitivas. Sempre que agimos com uma intenção
baixa, imediatamente nos tornamos um ser humano mais grosseiro. Sempre
que agimos com gentileza, imediatamente nos tornamos uma pessoa bastante
melhor.

\section{O que é que o Budismo diz sobre a reencarnação?}

Nas primeiras horas da noite em que Siddhattha Gotama se iluminou, ele
conseguiu vislumbrar um número considerável de vidas passadas. A meio da
noite foi capaz de seguir as caminhadas de seres em reinos diferentes,
ao longo de muitas vidas, e assim verificar a lei do kamma. Estas
experiências, tão inimaginavelmente intensas, minaram de tal maneira as
tão bem estabelecidas toxidades da sua mente, e em consequência
realçando tanto o poder destas contemplações, que, de manhã, ele se
tinha tornado um Buda completamente iluminado.

Ao longo da sua carreira de ensino o Buda revelou informações sobre
outros reinos. Em várias ocasiões falou destes diferentes reinos de
existência, bem como da conduta que levava a renascer neles. Parece
claro ele ter sentido que o conhecimento destes reinos concedia um
melhor contexto para o empenho espiritual. Mesmo que este conhecimento
não fosse verificável por experiência pessoal, considerou"-o um valioso
suporte para todos os que seguiam o Óctuplo Caminho.

O Buda esclareceu que nenhum reino é eterno, e que o renascer nos reinos
celestiais, independentemente de quão sublime seja, é sempre, em última
instância, insatisfatório, e tem sempre um término. Ele ensinou que o
ser que atinge a iluminação perfeita já não reencarna mais vez nenhuma.
A causa para o vagar sem começo nos reinos temporais é a ignorância
fundamental da natureza de como as coisas são. Uma vez essa ignorância
destruída, tudo quanto se baseia em tal, desaparece.

\section{Para os budistas, que importância tem acreditar na reencarnação?}

O Budismo não é um membro do sistema das `famílias com sistemas de
crenças' das religiões. Por essa razão os ensinamentos do Buda sobre
reencarnação não deveriam ser vistos como um dogma no qual os budistas
têm de acreditar. Os budistas são encorajados a assumir o ensinamento da
reencarnação como sendo fiável, mas a estar constantemente conscientes
que o facto de aceitarem um ensinamento que faz sentido, que inspira
confiança, ou que é tão consistente com outros ensinamentos já provados
como verdadeiros, não é o mesmo que conhecer a verdade por si próprio.

O Buda ensinou que as pessoas deviam `preocupar"-se com a verdade', não
reivindicando que algo tenha que ser necessariamente verdadeiro só por
que se tem uma forte sensação de que o é. A vasta maioria dos budistas
não foram efectivamente capazes de provar a verdade da reencarnação. São
ensinados a humildemente reconhecer que, de facto, não sabem se tal é
verdade, mas a aceitar os ensinamentos sobre reencarnação como uma
hipótese de trabalho para compreenderem as suas vidas, e para seguirem a
via do Buda para o despertar. Ao praticarem o Óctuplo Caminho, a
confiança no kamma e no renascimento cresce de uma forma natural, não
forçada.

\section{O que é que o Budismo ensina sobre o céu e o inferno?}

O céu e o inferno são considerados dois reinos da existência. O
nascimento em algum destes reinos ocorre como resultado da acção da
vontade. Embora a extensão de tempo de quem nasce num destes reinos seja
muito longa, acaba sempre por chegar ao fim. É por essa razão que se
considera não ser sábio o desejo de nascer no céu, após a morte. O céu é
uma pausa temporária dos rigores do nascimento, velhice, doença e morte,
não é uma libertação destes.

\section{Os budistas acreditam em espíritos?}

O Buda confirmou a presença, no mundo, de seres não humanos invisíveis a
olho nu. A existência desses seres tem vindo a ser verificada ao longo
dos anos, por médiuns dotados que desenvolveram as faculdades
necessárias para os perceber. A grande maioria dos budistas, que não
consegue confirmar a verdade nesta matéria, toma"-o como verdadeiro.
Outros, de disposições mais cépticas, têm reservas sobre tal.

Os professores budistas consideram que, mais importante do que estimular
a fé na existência de tais seres invisíveis, há que infundir atitudes
sábias para com eles. O Buda ensinou que todos os seres sem excepção são
nossos companheiros de caminhada nos reinos do nascimento e da morte, e
como tal não deveriam ser adorados nem subornados com ofertas. Os
budistas são ensinados a cultivar uma atitude de respeito e de bondade
para com os seres não humanos. Ao fazê"-lo, tornam"-se amados por eles, e
livres de quaisquer perigos que venham destes seres. E, se se der o caso
de os fenómenos percebidos como espíritos serem simplesmente produto do
inconsciente humano, tomar a mesma atitude é a melhor cura.

\section{Porque é que se dá tanta importância à impermanência no Budismo?}

A impermanência é a característica principal da existência. Tudo muda,
nada fica igual, nada dura para sempre. Embora isto possa parecer uma
observação banal, uma investigação mais atenta revela quantos dos
pensamentos, emoções, percepções, desejos e medos ocorrem precisamente,
porque a verdade da impermanência é constantemente esquecida. A reflexão
continuada sobre a condicionada e impermanente natureza das coisas evita
que nos entusiasmemos e nos descuidemos quando tudo corre bem, e que
fiquemos deprimidos e desencorajados, quando as coisas correm mal. Na
meditação, a mente centrada desenvolve intuição da sua verdadeira
natureza, através da observação, a cada momento, da ascensão e queda dos
fenómenos físicos e mentais.

\section{Qual é o significado do `não-eu'?}

A pessoa não iluminada assume que existe uma entidade permanente
independente que subjaz à experiência, e que esta entidade é o `nós',
quem nós verdadeiramente somos. Tomamos como garantido que este `eu' é
quem vê, pensa, sente, ouve, fala e age. O Buda ensinou que esta
compreensão de quem somos é enganosa, baseada em determinados erros de
percepção fundamentais, e é a causa"-raiz do sofrimento humano.

O Budismo ensina que, bem longe de ser o sólido centro de experiência, a
sensação do `eu' é criada a cada momento, usando uma identificação
instintiva com os aspectos da experiência -- o corpo, sentimentos,
percepções, pensamentos, emoções e autoconsciência. O Buda
encorajou"-nos a olhar mais de perto a nossa experiência, de forma a ver
se conseguimos descobrir este `eu', que parece existir de forma tão
óbvia. Ao reconhecer que a vida é um fluir de fenómenos, dependente de
causas e de condições, mas sem dono nem controlador, intui"-se o
`não"-eu', ou anattā.

Uma forma de compreender este ensinamento é a de considerar a frase:
`Chove' (em Inglês, `It rains'). Neste caso, a que é que se refere a
palavra `it' (N.T) nesta frase? Será que existe `it' que está a chover,
ou ao referirmo"-nos a `it' estamos simplesmente a empregar uma convenção
linguística?

O ensinamento do `não"-eu' é contra"-intuitivo e só consegue ser
realizado numa mente estável e feliz. Por esta razão, a ênfase é
colocada na criação de uma sólida base para esta intuição, através da
prática da generosidade, da conduta moral e da meditação.

N.T. - `It' é o 3º pronome pessoal, em inglês, usado para significar uma
coisa, animal, situação ou ideia que já foi mencionada anteriormente,
algo que não é usado em Português, uma vez que só temos masculino,
`ele', e feminino, `ela'. Curiosamente as expressões populares adoptam
algo semelhante ao `it', como por exemplo `ele hoje
chove' ou `ele vai chover'.

\section{Se não existe o eu, afinal o que é renascer?}

Os ensinamentos do `não"-eu' apontam para o facto de que as coisas
existem como um processo, mais do que como objectos distintos. Uma vela
fornece a analogia tradicional para ilustrar a relação entre o
`não"-eu' e o renascimento. Aquilo a que chamamos a chama de uma vela
não é, em si, uma coisa, mas a expressão da relação da ligação temporal
entre o pavio da vela e o oxigénio. Se uma vela for acesa a partir de
outra, é apenas convencionalmente verdade dizer que uma coisa chamada
chama migrou de uma vela para outra; efectivamente, um processo foi
mantido com o fornecimento de um novo material de base. Da mesma forma,
não existe algo chamado `eu' que tenha renascido pela morte de um
corpo, mas trata"-se antes de um processo que se manifesta de uma forma
nova e ajustada.

\section{Se não existe o `eu', como é que se pode responsabilizar as pessoas
  pelas suas acções?}

O Budismo faz uma distinção entre a realidade e a convenção social. A
ideia do eu é reconhecida como sendo um elemento da vida social muito
útil, até mesmo indispensável. Em conversas, os mestres iluminados usam
os termos `eu' e `tu' de uma forma normal e respondem por um nome. A
diferença é que eles reconhecem a convenção como sendo uma convenção, e
não confundem isso com a realidade última.

A maioria dos ensinamentos budistas lida com a vida sob uma perspectiva
convencional. O papel principal é dado à responsabilidade pessoal. No
\emph{Dhammapada} o Buda afirma:

\begin{verse}
  Na verdade cada um é o refúgio de si mesmo;\\
  Quem mais poderia ser o refúgio de cada um?\\
  Uma vez completamente dominado o seu eu,\\
  Obtém-se um refúgio, que dificilmente haverá melhor.

  Dhammapada 380
\end{verse}
