\chapterPhotoTwoPageLeft{buddha-moon-Gr-crop}

\setlength{\chapterTitleTopSkip}{10mm}
\definecolor{photoChapterText}{gray}{1}

\chapterNote{%
  O Tathāgata é o Puro, o Perfeitamente Iluminado\\
  Ele é impecável na conduta e na compreensão,\\
  O Conhecedor dos Mundos:\\
  Ele treina com perfeição todos aqueles que querem ser treinados;\\
  É o Professor de deuses e de humanos;\\
  É o Desperto e Santo.%
}

\chapter{O Buda}

\chapterPhotoTwoPageRight{buddha-moon-Gr-crop}

\section{Quem era o Buda?}

Há 2.600 anos nasceu uma criança na família real do clã Sakyan, um povo
que vivia no nordeste da Índia e que agora fica na fronteira do Nepal.
Foi"-lhe dado o nome de Siddhattha. Com 29 anos, o Príncipe Siddhattha
renunciou à vida de facilidades e privilégios em busca da libertação
espiritual. Seis anos depois, após uma memorável noite de meditação,
sentado de pernas cruzadas sob uma árvore bodhi, realizou `o
inexcedível despertar pleno'. Ao fazê"-lo, tornou"-se `O Buda', `O
Desperto'.

No seguimento do seu despertar, o Buda dedicou os restantes quarenta e
cinco anos de sua vida a revelar o Dhamma: a verdadeira realidade, bem
como o caminho conducente à realização dessa verdade. Durante esse
tempo, estabeleceu uma ordem monástica (Sangha) para os seus discípulos,
homens e mulheres, que queriam deixar as tarefas mundanas e devotarem"-se
com todo o seu ser ao estudo e à prática dos seus ensinamentos.

\section{O que é a iluminação?}

A iluminação refere"-se à libertação do sofrimento e das aflições mentais
ou `obstáculos' que são a sua causa. É a realização da própria
natureza de `como as coisas são'. Um ser iluminado compreende a
natureza condicionada dos fenómenos impermanentes e vivencia o
Nibbāna,\footnote{%
  `Nibbāna' em Pali é o mesmo que `Nirvana' em Sânscrito.}
a realidade incondicionada subjacente. O Buda referia"-se a este estado
como a `felicidade suprema'. A mente iluminada caracteriza"-se pela
sabedoria, compaixão e pureza. O Buda ensinou que todos os seres
humanos, masculinos e femininos, nascem com o potencial da iluminação.

O Buda falou dos quatro estádios de iluminação, e consequentes quatro
tipos de seres iluminados. O primeiro destes seres é `o que entra na
corrente', o segundo `o que volta uma vez', o terceiro `o que não
volta', e o último é o totalmente iluminado `o \emph{arahant}'. O
alcançar destes estados depende da prática do Óctuplo Caminho enunciado
pelo Buda. O seu resultado é assinalado pelo total desaparecimento na
mente de certos estados mentais confusos. Já não é possível regressar a
partir de tal estado. Aquele que alcança o primeiro estádio de
iluminação deve assegurar"-se de alcançar o estádio final no prazo máximo
de sete vidas. Ele, ou ela, entrou na corrente que conduz
irrevocavelmente ao oceano do Nibbāna.

\section{O que significa `Buda'?}

A palavra Buda significa `o que despertou'. O Buda ensinou que o ser
humano não iluminado vive num estado que pode ser comparado a estar a
dormir, ou a um sonho. Através da clara luz da sabedoria, e sem qualquer
ajuda, o Buda foi aquele que despertou desse sonho, para a verdadeira
natureza da existência. Guiado pela compaixão, o Buda é aquele que
procurou partilhar a sua compreensão da via do despertar, com todos os
seres que desejaram seguir as suas pisadas.

\section{O Buda era um ser humano?}

O Príncipe Siddhattha era um ser humano. Na noite em que realizou a
suprema iluminação, tornou"-se um Buda e, a partir desse momento, nunca
mais foi um ser humano, na acepção comum do termo. Para os olhos dos
não"-iniciados, o Buda poderá ter parecido como um forte líder religioso
carismático, alguém que teve uma morte normal aos oitenta anos. Contudo,
aqueles com faculdades mais desenvolvidas aperceberam"-se de que não
havia qualquer aparência externa, nem quaisquer palavras, conceitos, ou
categorias que servissem para \mbox{exprimir} a maravilhosa natureza imortal da
sua natureza de Buda.

\section{Que provas há da existência de Buda?}

\begin{enumerate}
\item
  Evidências arqueológicas fornecem fortes provas empíricas de Buda,
  enquanto figura histórica.
\item
  Muitos dos mosteiros e cidades mencionados nos discursos de Buda
  puderam ser localizados.
\item
  As relíquias de Buda foram recuperadas de locais mencionados nos
  textos.
\item
  O imperador budista Asoka, independentemente da data, esculpiu
  inscrições em colunas de grés que erigiu ao longo de todo o seu vasto
  império -- alguns dos quais sobreviveram até hoje -- referindo"-se
  extensivamente ao Buda.
\item
  Há muitas evidências circunstanciais nos primeiros textos.
\item
  A coesão e ausência de contradição interna nos discursos de Buda,
  juntamente com as prescrições finamente detalhadas para a ordenação do
  corpo monástico encontradas nos `Livros da Disciplina', apontam
  seriamente para um autor único.
\item
  Claro que a evidência física e a lógica sempre deixam espaço para a
  dúvida. Numa ocasião, o Buda disse: `Quem vê o Dhamma, vê"-me a mim'.
  Por outras palavras, a verificação da verdade dos ensinamentos na vida
  de cada um, é, sob o ponto de vista budista, a confirmação mais fiável
  da existência de Buda.
\end{enumerate}

\section{O Buda possuía poderes psíquicos?}

O Buda possuía imensos poderes psíquicos extraordinários. Os poderes
psíquicos podem (mas nem sempre) resultar de um treino intensivo da
mente e, ainda hoje, existem praticantes de meditação que possuem tais
poderes. O Buda usava os seus poderes psíquicos com moderação,
normalmente como auxílio nos ensinamentos, quando outros métodos
provavam ser ineficazes; o exemplo mais conhecido ocorreu no encontro
com o notório assassino, Angulimala. O Buda considerou que a fé das
pessoas obtida com a visão de `milagres' vulgarmente afastava"-as mais do
caminho da sabedoria, do que as aproximava. Por esta razão, proibiu os
monges com poderes psíquicos de os revelarem aos leigos. A posse de
poderes psíquicos pode"-se tornar viciante. O Buda recomendou que os seus
discípulos não os considerassem como fins em si na vida espiritual.

\section{Quantos Budas há?}

De acordo com a tradição Theravāda, só pode haver um Buda de cada vez.
Contudo, existiram outros Budas no passado longínquo, e existirão mais
futuramente. O intervalo entre a aparição dos Budas é medido em
\emph{kalpas}. Um \emph{kalpa} é uma medida de tempo extraordinariamente
longa. O Buda forneceu esta definição:

\begin{verse}
Supõe, bhikkhu, que existia uma enorme montanha de pedra com
dez milhas de comprimento (uma yojana), dez milhas de largura e dez
milhas de altura, sem buracos nem fendas, uma massa sólida de pedra. Ao
fim de cada cem anos, um homem bateria nela com um pedaço de pano
delicado. Essa enorme pedra poderá ser gasta e eliminada através desse
esforço, mas ainda assim o kalpa não terá chegado ao fim.

{\raggedleft
  Saṃyutta Nikāya, 15.5
\par}
\end{verse}

\section{Como era a relação de Buda com a sua família?}

O Buda demonstrou apreço pela sua família sob a forma que lhe era mais
adequada como Buda: conduzindo os seus membros para a via do despertar.
No primeiro ano após a sua iluminação, sete anos depois de se ter ido
embora, o Buda voltou à sua casa de origem na cidade de Kapilavatthu.
Esta visita viria a ter um profundo impacto, não apenas em todo o reino
Sakyan, mas mais ainda no seu pai, o rei Suddhodana; como resultado da
sua primeira visita, o rei realizou os dois primeiros níveis de
iluminação. Alguns anos mais tarde, o Buda, apercebendo"-se da aproximação
da morte de seu pai, visitou o velho rei pela última vez e conduziu"-o ao
estado de arahant, o estado de iluminação mais elevado. Esta visita a
Kapilavatthu foi também notável pelo primeiro encontro com o seu filho
de sete anos, Rāhula, durante o qual o jovem requereu a sua herança.
Como resposta, o Buda permitiu que ele se juntasse ao Sangha, como o
primeiro rapaz noviço.

O Buda não conseguiu ensinar a sua mãe em Kapilavatthu, uma vez que ela
tinha morrido ao dar à luz (a lenda conta que mais tarde ele a foi
ensinar no reino celestial onde ela residia); contudo, conseguiu ensinar
a sua madrasta e tia, Pajāpati. Foi ela quem requereu formalmente ao
Buda que fundasse uma ordem de monjas, e quando obteve o consentimento,
ela tornou"-se a sua líder mais sénior. A primeira geração de monjas
incluía muitas mulheres da família de Buda, incluindo a sua ex"-mulher
Yasodhara. Conta"-se que Pajāpati, Yasodhara e o filho de Buda, Rāhula,
todos eles atingiram o mais elevado estádio de iluminação.

\enlargethispage{\baselineskip}

Muitos dos parentes masculinos foram ordenados como monges e alguns
deles foram mencionados como tendo sido os seus discípulos mais
excepcionais. Entre eles estão Anuruddha, Nanda, e o mais famoso, o seu
companheiro de longa data, Ānanda.

\section{O Buda tinha sentido de humor?}

O Buda sabia que o sentido de humor, usado criteriosamente, pode levar à
verdade através de meios encantadores e que desarmam as pessoas. De vez
em quando, a sagacidade e o dom de discurso que o Buda tinha
desenvolvido ao longo da sua educação real vinham à tona nos seus
discursos com efeitos divertidos. Trocadilhos, reformulação brilhante de
termos, parábolas bizarras e analogias cómicas, podem ser encontrados
nos seus ensinamentos. Embora não exista nada em seus discursos que
evoque risos declarados no leitor moderno, ao ler algumas passagens,
poderão imaginar facilmente as faces do público de Buda engalanadas com
largos sorrisos.
