\chapterPhotoTwoPageLeft{sand-walk-Gr-crop}

\setlength{\chapterTitleTopSkip}{-5mm}
\definecolor{photoChapterText}{gray}{0}

\chapter{O Sangha}

\chapterPhotoTwoPageRight{sand-walk-Gr-crop}

\vspace*{90mm}

{\centering
\begin{tikzpicture}%
  \node[fill=white, fill opacity=0.5, text opacity=1]{%
    \begin{minipage}{0.9\linewidth}%
      \centering\normalsize\normalfont\itshape%
      \setlength{\parskip}{10pt}%
      \color{photoChapterText}%
      \vspace{0pt}%
      Eles são os discípulos do Excelso Buda, que praticaram bem\\
      Que praticaram directamente,\\
      Que praticaram de forma reflectida,\\
      Os que praticaram com integridade --\\
      São os quatro pares, os oito tipos de seres nobres --\\
      Estes são os Seus discípulos \\
      Tais seres são merecedores de dádivas,\\
      Merecedores de hospitalidade\\
      Merecedores de oferendas\\
      Merecedores de respeito;\\
      Dão ocasião a que, neste mundo, surja um bem incomparável.%
      \vspace{10pt}%
    \end{minipage}%
  };
\end{tikzpicture}%
\par}

\clearpage

\quoteTitleFmt{Dez assuntos, para lembrar frequentemente, por quem seguiu adiante}

\begin{verse}

Monges, existem dez dhammas sobre os quais se deve reflectir
frequentemente. Quais são estes dez dhammas?

Já não vivo segundo os valores e objectivos do mundo. 
Quem perfaz o caminho, deve reflectir sobre isto frequentemente.

A minha própria vida é sustentada pela generosidade dos outros. 
Quem perfaz o caminho, deve reflectir sobre isto frequentemente.

Devo esforçar"-me por abandonar os meus hábitos antigos. 
Quem perfaz o caminho, deve reflectir sobre isto frequentemente.

Surgem remorsos na minha mente em relação à minha conduta? 
Quem perfaz o caminho, deve reflectir sobre isto frequentemente.

Será que os meus companheiros espirituais acham falhas na minha conduta?
Quem perfaz o caminho, deve reflectir sobre isto frequentemente.

Tudo o que é meu, que amo e prezo, tornar"-se-á diferente,
separar"-se-á de mim. 
Quem perfaz o caminho, deve reflectir sobre isto frequentemente.

Sou dono do meu Kamma, herdeiro do meu kamma, nascido do meu
Kamma, ligado ao meu Kamma, permaneço suportado pelo meu Kamma;
seja qual Kamma eu criar, para o bem ou para o mal, disso serei o herdeiro. 
Quem perfaz o caminho, deve reflectir sobre isto frequentemente.

\enlargethispage{\baselineskip}

Os dias e as noites passam continuamente; como estou eu a usar o meu tempo? 
Quem perfaz o caminho, deve reflectir sobre isto frequentemente.

Aprecio a solidão ou não? 
Quem perfaz o caminho, deve reflectir sobre isto frequentemente.

Deu a minha prática frutos de compreensão e liberdade,
de forma a que no fim da minha vida eu não me sinta envergonhado, quando
questionado pelos meus companheiros espirituais? 
Quem perfaz o caminho, deve reflectir sobre isto frequentemente.

Monges, estes são dez dhammas sobre os quais se deve reflectir frequentemente.

{\raggedleft
Aṅguttara Nikāya, Livro dos Dez 48
\par}

\end{verse}

\section{O que quer dizer `Sangha'?}

A palavra `Sangha' é usada de duas maneiras. Primeiro, é o nome dado à
ordem monástica, daí a frase `O Sangha tailandês'. Segundo, refere"-se à
comunidade de todos aqueles que realizaram um dos níveis de iluminação.
As duas categorias sobrepõem"-se a um certo nível: durante os últimos
2.600 anos, a vasta maioria dos que realizaram níveis de iluminação
foram membros da ordem monástica. Contudo, a vida monástica não é
condição necessária para a iluminação. Muitos membros do Sangha
iluminado viveram (e vivem) como elementos de uma família.

\section{Porque é que os monges budistas rapam a cabeça?}

O cabelo é um dos maiores focos do desejo humano para embelezar o corpo
e projectar uma imagem particular ao mundo. Os monges rapam o cabelo
como uma expressão da sua aspiração à renúncia da vaidade pessoal. Ao
fazê"-lo, tal serve para lembrarem a si próprios e aos outros que
deixaram o mundo. A Visão de um monge budista em hábitos castanhos e com
cabeça rapada torna"-se memorável. As pessoas podem ficar curiosas ou
intrigadas, podem sentir"-se elevadas, podem ser lembradas da necessidade
de estar alerta e despertas. Por consequência, os monges budistas
propagam o Dhamma de uma forma muito delicada, basta serem vistos.

Os monges rapam a cabeça uma vez por mês (o dia anterior à lua cheia),
ou duas vezes (acrescentando o dia anterior à lua nova). Na Tailândia,
os monges também rapam as sobrancelhas.

\section{Porque é que os monges usam hábitos de cores diferentes?}

Os hábitos amarelo claro ou laranja são usados por monges que
vulgarmente vivem em mosteiros urbanos. Os castanhos são vulgarmente
usados por monge que vivem nos mosteiros das florestas.

Actualmente a maioria dos monges usa hábitos de material sintético.
Estes são produzidos comercialmente numa variedade de cores, comprados
por budistas leigos e depois oferecidos aos monges. Na maioria dos
mosteiros prescreve"-se uma série de cores, mas noutros os monges são
livres de usar qualquer cor que lhes seja oferecida, desde que esteja
dentro de limites aceitáveis.

Em muitos mosteiros da floresta os monges costuram os seus hábitos e
tingem"-nos com cor extraída do tronco da árvore da jaca. A cor destas
indumentárias varia de acordo com a cor da madeira usada (indo desde a
cor dourada até ao vermelho alaranjado), e com o uso do hábito (sempre
lavado numa solução diluída de tinta, um adstringente fraco, escurecendo consoante envelhece).

\section{O que é o Vinaya?}

O Vinaya é o compêndio das regras de prática, protocolos, procedimentos
e deveres deixados pelo Buda para a ordem monástica. O Vinaya tem como
intenção manter a harmonia dentro e entre as comunidades monásticas, bem
como criar as condições óptimas para cada monge praticar o Damma. O
cerne do Vinaya é o \emph{Pāṭimokkha}, as 227 regras que
constituem o código básico da disciplina.

O Phātimmokkha está dividido em várias secções. A primeira consiste em
quatro ofensas que dão origem a expulsão: as relações sexuais, roubar,
matar um qualquer ser humano, e fazer falsas alegações sobre realizações
espirituais. A segunda secção consiste em 13 ofensas extremamente
graves, as quais constituem sérias manchas na honra de um monge e, se
cometidas, requerem um período de penitência para se purificarem. Nelas
estão incluídas casos graves de má conduta sexual tais como masturbação,
tocar o corpo de uma mulher com luxúria, e seduzir sexualmente de forma
explícita. Todas as outras ofensas são confessadas e purificadas através
de um procedimento breve, precedendo a reunião quinzenal da comunidade
monástica, o \emph{Uposatha}, durante a qual toda a disciplina do
Phātimmokkha é cantada por um dos monges.

\section{O celibato monástico não será contranatura?}

Sim, se considerarmos que `contranatura' significa agir de formas que
ultrapassam os mais básicos instintos humanos. Mas é geralmente aceite
que a civilização humana tem vindo a evoluir até ao seu nível actual
precisamente pela sua capacidade de ir além do que lhe é inato, de
conseguir ser inteligentemente `contranatura'. Também se pode
argumentar que a aspiração humana de governar os instintos básicos é
concretizada na mente tão naturalmente, como os próprios instintos são
vividos no corpo. O impulso sexual é, talvez, o instinto mais forte, e
aprender a relacionar"-se com ele de uma forma hábil é um grande desafio.
No Sangha tailandês, os monges que não se acham capazes, ou não querem,
defender uma vida celibatária regressam à vida leiga sem qualquer
vergonha ou crítica.

\section{Qual é o verdadeiro objectivo do celibato monástico?}

O Sangha foi fundado pelo Buda para todos os que desejavam devotar"-se
com afinco à sua via para o despertar. O Buda planificou a vida
monástica como sendo de uma simplicidade radical, com o mínimo de
distracções desnecessárias. As ligações românticas, as relações sexuais
e suas consequências naturais -- a parentalidade -- são incompatíveis
com a prática criada. Igualmente comprometeriam a relação simbiótica
entre a ordem mendicante e a sociedade em geral, a qual o Buda
vislumbrou.

O Buda descobriu que as formas mais subtis de felicidade e a experiência
do verdadeiro bem"-estar são raramente acessíveis e sempre
insustentáveis, sempre que as pessoas se entregam aos prazeres
sensoriais. Ele insistiu no celibato do Sangha de forma a conceder aos
monges a oportunidade de investigarem o impulso sexual como um fenómeno
condicionado e aprenderem a se libertar da identificação com ele.

O Buda revelou que com a maturidade espiritual o instinto sexual não tem
sustentáculo e desvanece. Tal como os desejos sexuais, as percepções e
os pensamentos não são só um obstáculo à libertação, mas devem a sua
existência à ignorância profundamente enraizada sobre a verdadeira
realidade, a que os monásticos se dedicam a eliminar e, daí, escolherem a
vida do celibato.

\section{Qual a finalidade da ronda esmoler?}

Os budistas consideram que o trabalho dos monges (o estudo, a prática e
o ensinamento do Dhamma) é tão importante, que deveriam estar livres
para o concretizar sem quaisquer preocupações de necessidades básicas
materiais. As famílias acreditam obter muito mérito sempre que suportam
materialmente o Sangha.

O Buda projectou a disciplina monástica de forma a prevenir que os
monges se desligassem completamente do mundo. As regras de prática que
se relacionam com a comida são as que desempenham um maior papel na
obtenção deste objectivo. Uma regra, por exemplo, estipula que os monges
só podem comer a comida que lhes tenha sido formalmente oferecida pelos
leigos budistas na manhã a ser comida. Isto assegura o contacto diário
entre os leigos e os monges, significando que até o mosteiro mais remoto
na floresta deve estar num espaço próximo de uma aldeia, de tal forma
que se possa ir a pé. A ronda esmoler é uma expressão diária da relação
simbiótica entre o Sangha e a comunidade leiga budista. Por irem à
aldeia local, os monges recebem o seu sustento diário, e os leigos, no
acto de dádiva, são relembrados da moral e dos valores espirituais.

A ronda esmoler tem um benefício espiritual tanto para os monges como
para os leigos. Para os monges sinceros, é um relembrar humilde e
frequentemente comovente da generosidade que lhes permite levar uma vida
monástica. Inspira"-os a expressar o seu apreço pela fé que lhes é
dedicada por serem diligentes na prática dos seus deveres. A ronda
esmoler dá aos leigos budistas a oportunidade de começarem o dia com um
acto de generosidade. Sentem a alegria de dar e o contentamento de terem
contribuído para o bem"-estar dos monges cuidados por eles. Também é uma
oportunidade para poderem dedicar aos seus entes queridos falecidos o
mérito que obtêm da sua generosidade. Muitos pais ensinam os filhos,
desde tenra idade, a pôr comida nas malgas dos monges, iniciando as
crianças numa actividade com os monges que consideram especial, bem como
criadora de um sentido de familiaridade e de conexão.

\section{Porque é que o Buda consentiu que os monges comessem carne?}

A primeira razão e a mais importante é que comer carne não é, em si,
considerado censurável. O Buda consentiu que os monges comessem carne,
caso não tivessem visto, ouvido, ou suspeitado que quaisquer seres vivos
tivessem sido mortos especificamente para fazer a refeição para eles. Em
tal caso, não tendo contribuído de forma directa para a morte das
criaturas, os monges não criavam kamma por consumirem a sua carne. Tão
pouco o Buda proibiu que os monges praticassem o vegetarianismo, nem o
louvou. Os seus ensinamentos sobre comida focaram"-se na importância de
se comer com moderação comida facilmente digerível, mais do que
defenderem qualquer dieta em particular.

Uma segunda consideração subjacente à atitude do Buda para com o
vegetarianismo no Sangha é o bem"-estar a longo prazo na própria ordem.
Os monges são mendicantes, dependendo totalmente da generosidade das
famílias nas necessidades alimentares; não lhes é permitido cultivar,
armazenar ou cozinhar comida, nem podem colher frutos das árvores. Se o
Sangha se tornasse restrito em áreas nas quais dependesse de doadores
vegetarianos, a sua influência benéfica na sociedade seria
desnecessariamente limitada. O espírito de mendicância também seria
traído, caso os monges pedissem comida especial aos doadores, mais do
que estarem gratos por qualquer oferta, feita de boa fé.

\section{Para atingir a iluminação é preciso aderir a uma ordem monástica?}

O Sangha foi criado pelo Buda especificamente de forma a providenciar as
condições óptimas para os homens e mulheres que quisessem comprometer"-se
verdadeiramente com a via do despertar. Por esta razão, o Sangha é a
vocação que dá mais apoio para os que, seriamente, querem praticar o
Budismo. Contudo, a vida monástica não serve para todos e mesmo muitas
pessoas que encaram a prática budista com seriedade têm obrigações que
não possibilitam a ordenação. Felizmente, para aqueles que não querem,
ou não podem levar uma vida monástica, seguir o caminho para a
iluminação numa família, embora difícil, pode conduzir a uma conclusão
satisfatória. Ao longo dos últimos séculos, muitos budistas leigos
levaram vidas exemplares e até alcançaram estádios de iluminação,
particularmente do primeiro nível, conhecidos como a `Entrada na
Corrente'.

\section{O que quer dizer `tudong'? O que é um `monge em tudong'?}

O termo `tudong' deriva do termo Pāli `dhutanga' referindo"-se a
treze práticas consentidas ao Sangha pelo Buda que `vão contra a
corrente'. Esta lista de práticas ascéticas incluem comer uma refeição
por dia, comer somente o que é colocado dentro da malga e viver junto à
raiz de uma árvore, terminando com a prática mais exigente: abster"-se da
postura deitada. As práticas do tudong desempenham um papel proeminente
nos mosteiros da floresta do nordeste da Tailândia e muitas estão
inseridas na vida diária das comunidades monásticas. Os monges partem
para determinadas práticas de tudong, por períodos limitados, de forma a
saírem fora da `zona de conforto' e para energizarem as mentes quando
são apanhados na rotina.

Fora das comunidades monásticas a palavra `tudong' é frequentemente
usada referindo"-se à prática em que os monges andam pelo campo, passando
as noites debaixo das suas redes mosquiteiras (\emph{glots}). Os monges
em tudong por vezes decidem deslocar"-se de um mosteiro para outro;
outras vezes escolhem uma rota que lhes permita visitar professores
afamados de forma a pedirem conselhos e encorajamento. Muitos procuram
as áreas remotas de forma a testarem"-se em ambientes menos familiares e
desconfortáveis, enfrentando os medos dos espíritos e dos animais
selvagens, meditando na solidão das montanhas e das cavernas.

\section{Os monges assumem os votos para sempre?}

Entrar numa ordem monástica implica comprometer"-se com a prática
monástica que lhes é proposta durante o tempo que entende ser necessário.
Um monge pode assumir para consigo próprio o voto de permanecer monge
para o resto da vida, mas tal não lhe é exigido. Na verdade, a maioria
dos que entram na ordem acabam por a deixar.

A ordenação temporária tem sido, desde há muito tempo, uma
característica chave do Budismo Tailandês. Tradicionalmente, os rapazes
ordenam"-se por três meses no retiro da estação das chuvas
(\emph{vassa}), o qual acontece entre as luas cheias de Julho e de
Outubro. O valor deste costume reside, em primeiro lugar, na
possibilidade de um jovem receber uma imersão nos valores morais e
espirituais, antes de se comprometer nos desafios do casamento e da
carreira. Em segundo lugar, fornece"-lhes uma forma de poderem exprimir
gratidão aos pais pela educação dada (acredita"-se que, através das suas
ordenações, os pais obtêm grandes méritos). Em terceiro lugar, este
hábito cria laços entre os budistas leigos e os mosteiros (onde, quer
eles, quer a família, foram monges) que perduram por gerações.

Os monges que se juntaram ao Sangha pretendendo permanecer aí o resto
das suas vidas, frequentemente sentem ser mais difícil do que tinham
imaginado e, após algum tempo, começa a surgir de novo o desejo da vida
leiga. Os professores geralmente aconselham os monges que estão a
considerar deixar o hábito a esperarem algum tempo, antes de tomarem uma
decisão firme, para verem se as suas intenções mudam. Mas, se um monge
decidir deixar a ordem, não fica sujeito a qualquer estigma social. Pelo
contrário, as comunidades leigas budistas geralmente têm um respeito e
uma confiança especial pelos homens que passaram uma parte das suas
vidas como monges.

\section{O que é que os monges fazem diariamente?}

A vida diária dos monges depende do tipo de mosteiro onde vivem e do
nível da sua carreira monástica. Nos mosteiros situados nas aldeias,
vilas e cidades da Tailândia, os monges assistem aos serviços matinais e
vespertinos, saem para a ronda esmoler de manhã cedo, e passam o resto
do dia a estudar, ensinar ou a realizar deveres cerimoniais. Falando em
termos gerais, a prática da meditação não constitui a maior parte de
suas vidas. Nestes mosteiros os monges comem duas vezes por dia: a
primeira refeição depois da ronda esmoler e a segunda por volta das
onze da manhã.

Nos mosteiros da floresta os monges levantam"-se por volta das três da
manhã. Em alguns mosteiros, os cânticos de grupo e a meditação acontecem
de manhã cedo e à noite; noutros, os monges meditam sozinhos. Logo ao
raiar do sol, os monges põem"-se a caminho para a ronda esmoler em
direcção aos vilarejos das redondezas, em itinerários que distanciam
entre, aproximadamente, dois a dez quilómetros. Os monges da floresta só
comem uma vez por dia, geralmente por volta das oito da manhã. Passam
muito do dia a praticar meditação, sentada e a andar. O estudo dos
livros tem um papel secundário e é deixado à escolha de cada um. Pode
ser que recebam instrução formal do seu professor, duas a quatro vezes
por mês. De tarde, os monges trabalham geralmente uma ou duas horas,
principalmente na limpeza dos edifícios monásticos e a varrer os
caminhos da floresta. Nos mosteiros mais pobres, é frequente os monges
fazerem algumas obras de construção de que precisam.

\section{Uma vez que os monges vivem em reclusão, que qualificações têm para
  aconselhar as pessoas sobre as suas famílias e seus problemas profissionais?}

Pessoas de qualquer classe e nível social, velhos e novos, do sexo
masculino e feminino, vão visitar os monges seniores. Elas tratam das
suas vidas e problemas com estes monges, da mesma forma que os
ocidentais falam com um padre ou um terapeuta. Em resultado disso, esses
monges acabam por ter uma boa visão dos diversos problemas com que se
confrontam os seus discípulos leigos.

Uma vida devotada à compreensão da mente humana significa que os monges
séniores, particularmente os mestres em meditação, alcançaram profundas
realizações no que respeita à forma como a mente funciona, como cria
sofrimento e como se pode libertar de tal. Ao observarem profundamente o
funcionamento das próprias mentes, estes monges compreendem a mente dos
outros. Embora as situações que provocam emoções possam variar, as
emoções em si são universais. Ao falarem sobre os pensamentos, as
crenças, os desejos e medos que subjazem a vários problemas, os monges
podem chegar à raiz do assunto em causa, sem se atrapalharem por falta
de experiência pessoal em situações particulares.

\section{As mulheres podem ser monjas?}

Sim, as mulheres podem ter uma vida monástica, dedicando as suas vidas
ao estudo e prática dos ensinamentos budistas, mas não da mesma maneira
que as monjas das primeiras gerações. Infelizmente, a ordem original das
monjas, o Sangha das Bhikkhunīs, extinguiu"-se há mais de mil anos. A
visão que prevaleceu (embora não unânime) dos países budistas Theravāda
é que não é possível restaurar a ordem das Bhikkhunīs, uma vez que os
requisitos para a ordenação das Bhikkhunīs tal como o Buda estipulou não
podem ser preenchidos. Sendo a linha Theravāda uma tradição que se
define pela atitude conservadora para com os textos, não é de
surpreender que a atitude de ultrapassar as instruções dadas pelo
próprio Buda seja considerado impensável para muitos monges. Como
alternativa para as ordenações Bhikkhunīs, os países Theraváda
estabeleceram as suas próprias instituições quase monásticas para
mulheres com fé. Na Tailândia esta instituição é a ordem dos hábitos
brancos das \emph{mae chee}.

A visão ortodoxa sobre a restauração das ordens Bhikkhunīs não é aceite
universalmente. Ao longo dos últimos anos, iniciou"-se um movimento para
reestabelecer a ordem Bhikkhunī, principalmente instigado por mulheres
budistas de países ocidentais. Há um número crescente de mulheres que,
actualmente, vivem como Bhikkhunīs, sendo que um pequeno número delas
reside na Tailândia.

\section{Alguma vez houve na Tailândia uma Ordem de Bhikkhunīs?}

Na altura em que o primeiro reino tailandês se instalou em Sukhotai, no
séc. XII, a ordem das Bhikkhunīs já tinha sido extinta. É possível que
aproximadamente mil anos antes disso, na primeira vaga da propagação
budista, no que hoje é o centro da Tailândia, as Bhikkhunīs tenham
desempenhado algum papel, mas não há relatos históricos fiáveis de tal.
Seguramente não existe qualquer memória cultural de uma `idade de
ouro' budista, em que o Sangha fosse constituído de ambos: Bhikkhus e
Bhikkhunīs. Talvez isto ajude a explicar porque é que a atitude
tailandesa perante a restauração das ordens Bhikkhunīs seja pouco
entusiasta.

\section{Não é melhor trabalhar para tornar a sociedade um lugar melhor, em vez
  de se retirar para ser um monge ou uma monja?}

A ordem monástica está ligada à sociedade em geral através de uma
complexa rede de relações. No cômputo geral, deve ser considerada mais
como uma parte especial da sociedade, do que um corpo exterior a ela.
Tornar"-se monge não significa cortar todos os laços com o mundo, mas
antes adoptar uma nova relação com ele. Os monásticos defendem a
integridade dos ensinamentos do Buda, evitando que se diluam ou sejam
distorcidos. Põem em prática os ensinamentos e, ao longo das suas vidas,
tentam comprovar o valor de o fazerem. Os monges transmitem os
ensinamentos aos budistas leigos. Empreendem uma prática que, se feita
com sinceridade, os ajuda a moldar as qualidades da paz, compaixão e
sabedoria. Ao desempenharem estas funções, acredita"-se que os monásticos
ajudam a preservar e disseminar os valores que corroboram nas tentativas
de fazer mudanças positivas na sociedade.

\section{Porque é que na Tailândia os monges recebem, das mulheres, oferendas
  sobre um pedaço de tecido, em vez de receberem directamente das suas mãos?}

A prática não foi proposta pelo Buda, e não é seguida por outros monges
em outros países Theravāda. É uma convenção (possivelmente copiada de
algum ritual da corte Khmer) que foi adoptada pelo Sangha tailandês há
uns séculos atrás, de forma a manter a formalidade de relações entre
monges e mulheres leigas. O acto de doar cria, pela sua natureza, uma
certa intimidade entre aquele que dá e o que recebe. Ao aceitar ofertas
num pano, o monge cria uma artificialidade e distância no acto de doar,
servindo de ajuda à consciência, bem como uma restrição, tanto para o
monge, como para a mulher que dá.
