\chapterPhotoTwoPageLeft{drops-Gr-crop}

\setlength{\chapterTitleTopSkip}{65mm}
\definecolor{photoChapterText}{gray}{1}

\chapterNote{%
  Todas as acções são ditadas pela mente:\\
  A mente é o seu mestre; a mente é o seu criador.\\
  Se alguém agir ou falar com um estado mental contaminado,\\
  O sofrimento surgirá, tal como o carro de rodas\\
  Que segue as pegadas do boi.

  Todas as acções são ditadas pela mente:\\
  A mente é o seu mestre; a mente é o seu criador.\\
  Se alguém agir ou falar com um estado mental puro\\
  Surgirá a felicidade, como uma sombra\\
  Que perdura por detrás, sem partir

  Dhammapada, v. 1-2
}

\enlargethispage{2\baselineskip}

\chapter{Atitudes Budistas}

\chapterPhotoTwoPageRight{drops-Gr-crop}

\section{Porque é que se derramou tanto sangue em nome da religião? O~Budismo
  também contribuiu para essa carnificina?}

\vspace*{-\baselineskip}

Os seres humanos precisam de dar sentido às suas vidas, dar"-lhes
significado e propósito. As religiões surgiram para ir ao encontro desta
necessidade. Como consequência, a maioria das pessoas adopta um tipo de
crenças ou dogmas religiosos como um enquadramento dentro do qual
conseguem compreender as suas vidas. Mas, como há um número de sistemas
de crenças, e cada um tem tendência a afirmar que só ele é que é o
possuidor da verdade, o ancestral conflito, entre os sistemas de crenças
e as paixões que engendram, é inevitável. Apesar de toda a violência
`religiosa' que o mundo já presenciou, a natureza das necessidades
humanas faz com que se torne difícil imaginar um mundo sem pessoas a
refugiarem"-se nas crenças dogmáticas.

Até mesmo as pessoas assumidamente seculares ou materialistas têm as
mesmas necessidades psicológicas de estabilidade e de significado. Na
verdade, conseguem identificar"-se com as suas próprias visões e crenças
com tanta tenacidade como os mais convencionais dos religiosos. Nos dias
de hoje, não é difícil depararmo"-nos com exemplos de crenças políticas,
e até de teorias científicas, que caem nas malhas do dogma.

Os budistas orgulham"-se de que, nos seus textos, não exista uma única
frase que possa justificar o derrame de uma só gota de sangue. Em alguns
lugares, contudo, os textos ficam por ler e as lições por ensinar. No
mundo actual, uma pequena minoria de homens que envergam hábitos de
monges budistas usam a sua autoridade para agravar, mais do que aplacar,
as disputas étnicas e territoriais, enquanto o resto do mundo Theravāda
observa consternado.

Ainda assim, os ensinamentos de Buda oferecem seguramente um caminho
para sair da violência religiosa. Asseguram que a via para a segurança e
significado real reside nas acções do corpo, da fala e da mente, mais do
que na crença. A fé reside na capacidade de mudar, que devemos pôr à
prova, ao contrário do dogma. No esforço de educar os seus
comportamentos, emoções e compreensões, os seres humanos podem encontrar
uma finalidade que não crie um sentido alienado de todos aqueles que não
partilham do seu compromisso.

\section{Qual é a atitude do Budismo para com as mulheres?}

Primeiro e o mais importante, as mulheres são vistas como seres humanos
sujeitos a nascer, envelhecer, adoecer e a morrer: seres propensos ao
sofrimento e com a capacidade e oportunidade de o transcender. O Buda
deixou bem claro que a capacidade para a iluminação não se baseia no
género. Está presente na virtude de um nascimento humano e as mulheres
são vistas como tendo o mesmo potencial espiritual que os homens.

À luz desta visão de capacidades espirituais, o Buda deu às mulheres,
que se queriam devotar inteiramente à sua prática, a oportunidade de se
tornarem monjas. Passou muito tempo a ensinar as mulheres, tanto as
monjas como as leigas, não lhes restringindo qualquer ensinamento por
causa do seu sexo.

Mas a asserção do Buda sobre a igualdade espiritual das mulheres não o
levou a defender mudanças radicais na ordem social dominada pelos
homens. Reservou a sua crítica social para o que lhe parecia ser a
característica mais perniciosa: o sistema de castas. Curiosamente, no
Sangha, a área na qual o Buda tinha efectivamente o poder de estabelecer
convenções que governassem as relações entre homens e mulheres, ele não
optou pela igualdade. O Buda estabeleceu relações entre as duas ordens
monásticas para que a ordem das monjas, instaurada depois da dos monges,
fosse considerada a sua irmã mais nova. O Buda percebeu que este tipo de
hierarquia leve, protegida de abusos, por validação e normas de
equilíbrio incluídas na disciplina do Vinaya, era a melhor maneira de
gerir as comunidades renunciantes, e a mais aceitável para a sociedade
em geral.

\section{O que é que o Budismo pode dizer sobre a sexualidade humana?}

\enlargethispage{\baselineskip}

Como a sexualidade humana é uma força tão poderosa e tão potencialmente
tumultuosa nas relações humanas, o Buda ensinou que ela precisa de ser
gerida com sabedoria. Os budistas leigos assumem um preceito moral que
lhes propõe refrear de todas as formas de sexualidade ilícita; o
preceito enfatiza o adultério, mas inclui a violação e o assédio sexual.
Os budistas são encorajados a cuidar dos seus sentidos e a não se
entregarem à visualização, audição, cheiro, paladar e toque de tudo
quanto estimule demasiado os apetites sexuais. O desejo por prazer
sexual pode conduzir, quando consentido de forma cega, a muito
sofrimento, e, em alguns casos, pode resultar na traição da confiança,
destruição de famílias, ruína financeira ou em actos de violência. As
pessoas sensatas vêem os prazeres sexuais como algo fortemente viciante
e valorizam a sua liberdade o suficiente para evitarem que o sexo lhes
monopolize a vida, influenciando insensatamente as escolhas a fazer.

O desejo sexual não é visto como um mal em si. Contudo, é visto como
sendo condicionado pela ignorância da verdadeira forma de ser das
coisas. Por esta razão, os que meditam são encorajados a investigar a
sua natureza. O Buda, numa das suas análises sobre o anseio sexual,
explica até que ponto é que a obsessão de qualquer mulher com a sua
condição de feminilidade se relaciona com o grau de atracção que exerce
no masculino, bem como qual o grau em que a obsessão de qualquer homem
com a sua condição de masculinidade influencia na atracção pelo
feminino.

Os ensinamentos budistas não vêem como inerentemente más as relações
consentidas entre seres adultos do mesmo sexo. As sábias reflexões do
Buda sobre o desejo sexual são igualmente aplicadas independentemente do
género do objecto a que se destina o desejo.

O desejo sexual diminui de forma significativa com o Óctuplo Caminho,
consoante o praticante começa a experimentar outras formas de prazer e
de libertação mais satisfatórias. Os sentimentos universais de amor e
compaixão sobrepõem"-se ao desejo de intimidade pessoal. O forte impulso
que antes impelia para a actividade sexual pode então ser reconhecido,
em parte, como uma deslocação de actividade, alimentada por uma
incapacidade de o ver tal como é: esse tão fortemente enraizado desejo
de libertação interior. O `arahant' perfeitamente iluminado não
experiencia qualquer desejo sexual, e, contudo, vive no supremo e
inabalável bem"-estar.

\section{Como é que o Budismo encara o amor?}

Nos ensinamentos budistas o amor é visto como uma benesse, ou como uma
toxidade dos estados mentais presentes na mente dos amantes e dos
amados. Várias emoções podem ser enunciadas. Ao nível mais básico, o
amor pode ser narcisístico e exigente; sob a perspectiva mais sublime, o
amor é altruísta e incondicionado. O amor pessoal tende a oscilar entre
estes dois extremos. Os budistas são ensinados que, quanto mais o seu
amor se inclinar para as formas egoístas, mais sofrerão e mais
sofrimentos causarão aos entes queridos; quanto mais incondicional for o
amor, e mais fortemente baseado na sabedoria e na compreensão, maior
felicidade sentirão e maior capacidade terão de dar felicidade aos
outros. Os budistas são ensinados a cultivar os actos, as palavras, os
pensamentos e as emoções de forma a educarem e purificarem as emoções
positivas.

\section{Qual é a atitude do Budismo para com as outras religiões?}

O Buda enalteceu os elementos de outras religiões que se harmonizavam
com o seu caminho para o despertar; criticou as crenças e práticas que
incrementavam a dimensão da superstição, da crueldade e do dano no
mundo. Defendeu a boa vontade e o respeito por todos os seres vivos,
independentemente das suas crenças.

A intolerância religiosa é alheia ao Budismo Theravada. Significativo é
o facto de não haver na língua tailandesa uma palavra que possa exprimir
tal conceito. Como o Budismo não considera que exista qualquer salvação
que dependa de uma crença particular num conjunto de dogmas, não vê que
uma diversidade de crenças seja ofensiva, e não faz conversões. Na
verdade, a disciplina monástica proíbe que os monges ensinem quem quer
que seja, incluindo budistas leigos, sem que primeiro lhes tenha sido
dirigido o pedido de o fazer.

À parte de casos ocasionais de evangelização contra a ética do Budismo
na Tailândia, a descriminação de budistas contra membros de outras
religiões é virtualmente inaudita. Embora os militantes muçulmanos
venham alvejando os budistas no sul do país há muito tempo, não tem
havido repercussões contra as comunidades muçulmanas em outras partes do
país. Por vezes, os princípios budistas podem"-se ser difíceis de
reconhecer na sociedade tailandesa contemporânea, mas uma atitude
amadurecida para com as outras religiões é uma luz viva que remanesce
brilhante.

\section{O Buda tinha algumas ideias sobre os assuntos económicos?}

Ao incluir o `Meio de Subsistência Correcta' no Óctuplo Caminho, o
Buda reconheceu o papel da actividade económica, quer na promoção do
bem"-estar individual, quer no desenvolvimento da sociedade em harmonia
com os princípios do Dhamma. Ensinou que os budistas deveriam ter em
conta critérios morais e espirituais ao considerarem o seu meio de
subsistência, em particular abstendo"-se de formas que pudessem
prejudicar outras pessoas, os animais ou o ambiente.

O Buda enfatizou a importância de viver a vida de forma honesta e com
uma motivação saudável. Sublinhou o quanto a honestidade conduz ao
auto"-respeito e como ajuda a criar uma atmosfera de confiança mútua no
local de trabalho (o que foi apontado pelos economistas budistas actuais
como um factor que leva a substanciais reduções de custos nas
transacções). Quando o desejo se foca nas remunerações do trabalho, mais
do que no prazer do trabalho bem feito, é bem provável que surja uma
forma de pensar a curto prazo e corrupção. Se a mente das pessoas se
focar mais na qualidade do trabalho em si, do que nas compensações
materiais, elas vivem mais contentes, menos stressadas e fazem um
trabalho melhor.

Um tipo de atitude económica que o Buda frequentemente criticava era a
acumulação de riqueza. Afirmava que, embora o gasto não devesse exceder
o rendimento, as pessoas sensatas usam a sua riqueza para o seu
bem"-estar e de suas famílias; são generosos com os familiares e amigos,
e oferecem apoio ao Sangha e aos necessitados. O Buda mencionou a
comida, a roupa, o abrigo, e os remédios, como sendo os quatro
requisitos indispensáveis para uma vida sustentável. A ausência de algum
destes requisitos -- ou viver em ansiedade constante no receio de os
perder -- é um causa importante do sofrimento humano e torna
praticamente impossível a prática espiritual. Para casos em que as
condições locais não possibilitem a obtenção dos quatro requisitos, o
Buda ensinou que o governante ou o governo deveria oferecer assistência.
Em termos budistas, a medida de uma economia não se encontra no número
de milionários que consegue produzir, mas no grau em que consegue
garantir acesso, para todos, aos quatro requisitos.

\section{Qual a atitude do Buda relativamente à política?}

O Buda manteve uma postura neutra relativamente aos assuntos políticos.
Não tomou qualquer posição no que toca a programas políticos em
particular, e não defendeu nenhum grupo político na sociedade em
detrimento de outro. Embora evitasse falar a favor de qualquer forma de
governo em especial, efectivamente falou dos princípios gerais da
governança sábia e das virtudes e responsabilidades de quem estivesse no
poder. Ensinou quais deveriam ser os princípios para guiar os grandes
monarcas, bem como para os que asseguravam uma república saudável.

\section{Que atitude tomou o Buda no que respeita às diversões, tais como o
  cinema e os desportos?}

O Buda ensinou os seus discípulos a considerarem de que forma as suas
actividades reforçam o caminho para o despertar, ou os afastam de tal.
Estabeleceu um princípio geral: qualquer actividade que aumente o
domínio das actividades nocivas no coração e que diminua a força das
qualidades enriquecedoras no mesmo, deve ser evitada; qualquer
actividade que aumente a intensidade das qualidades enriquecedoras no
coração e que diminua a força das qualidades nocivas deve ser exercida
atentamente. Este é o princípio aplicado para determinar a correcta
relação para com todas as formas de entretenimento, desde a mais
grosseira à mais refinada.

O Buda reconheceu a necessidade que, neste mundo, as pessoas com vidas
estressadas, têm de se descontrair e de entretenimento. Por esta razão,
não encorajou os budistas leigos a absterem"-se de tais prazeres
completamente. Não obstante, recomendou um dia de abstinência, duas
vezes por mês, (nos dias de lua cheia e de lua nova). Aparte de deixar
mais tempo para as práticas religiosas, estes dias permitem que as
famílias se afastem das obrigações diárias e reavaliem de que forma as
suas vidas estão em harmonia com os seus fins e aspirações.

\section{Será que a criação e a fruição da arte podem ser consideradas um
  caminho espiritual?}

Sim, mas na perspectiva budista os seus benefícios são relativamente
superficiais. A arte maior pode elevar a mente e iluminar a condição
humana, de forma profunda e emocionalmente satisfatória, mas enferma em
si mesma na capacidade de induzir à duradoura transformação da
consciência, a qual é obtida pela prática do Óctuplo Caminho. Ainda
assim, quanto ao nível da produção e do enriquecimento que a arte
promove no cultivo de estados mentais sustentáveis, tais como estar
consciente e ser capaz de reflectir sobre si próprio, pode ser vista
como um suporte no caminho do despertar.

\section{O Budismo ensina o contentamento. Mas se toda a gente estivesse
  contente com a sua vida, como é que se alcançaria o progresso humano?}

As virtudes ensinadas pelo Buda são para serem entendidas no contexto
geral do caminho que o levou ao despertar. Sempre que o Buda fala de
contentamento, coloca"-o ao lado de uma característica energética como a
diligência, a persistência e a engenhosidade. Ele foi muito cuidadoso ao
deixar claro que o contentamento nada tem que ver com preguiça, nem
serve para significar passividade. O contentamento, no sentido budista,
deve ser apreciado à luz da importância central que o Buda dava ao
esforço humano. O Buda criticava claramente as filosofias que promoviam
o fatalismo, e uma vez até comparou as pessoas negligentes a corpos
deambulantes. O contentamento não anula o esforço, mas reforça qual a
base mais sólida onde deve ser praticado.

As pessoas não iluminadas sentem frequentemente que lhes falta algo, que
aquilo que não possuem os fará mais felizes do que aquilo que já têm.
Até quando o desejo é realizado, e a mente saciada, mas o sentido de
carência não alterado pela experiência, esta esperança sobrevive. Ao
aprendermos a apreciar o mérito que existe naquilo que possuímos,
libertamo"-nos de anseios, frustrações e invejas. Criamos objectivos
realistas, e com diligência aplicamo"-nos de forma a criar as causas e as
condições para a concretização desses objectivos. Mas entretanto,
apreciamos, tanto quanto possível, a situação presente. Seria algo muito
triste colocar todas as esperanças num futuro que pode nunca chegar.

\section{O que é que o Budismo ensina relativamente à nossa relação como o
  ambiente?}

\enlargethispage{\baselineskip}

O Buda tinha uma memória assombrosa das vidas passadas, e embora pudesse
lembrar"-se, literalmente, de `éons de contracção e expansão
universal', declarava não saber onde começava este `preambular'. Como
consequência, o Budismo não subscreve a ideia de que este mundo é o
trabalho de um deus criador e não atribui qualquer importância teológica
a este mundo natural onde vivemos --é visto como um fenómeno que existe
no presente estado devido a um fluxo de causas e condições. O desafio
que nos é colocado como espécie é o de sermos capazes de lidar com o
mundo físico de uma forma que fortaleça o mais possível a capacidade de
nos sustentar a nós próprios.

O Budismo defende que, para que possamos reforçar melhor a nossa relação
com o mundo físico, a educação deve actuar em três áreas: a conduta, a
emoção e o intelecto.

A educação da conduta requer que coloquemos o bem"-estar do planeta acima
das necessidades económicas a curto prazo - o que significa que devemos
refrear a vontade no que respeita a certos tipos de actividades
prejudiciais, adoptando uma forma de vida mais simples, com menos
desperdício. As mudanças exigidas ao nível da conduta não podem ser
levadas a cabo apenas por uma elite educada; para terem êxito têm de ser
adoptadas por todos. Por esta razão devem ser apoiadas por leis, normas
usuais e culturais.

A educação das emoções requer que inculquemos, nas nossas culturas e
dentro de cada coração, o amor e o respeito pelo mundo natural,
levando"-nos a repudiar a destruição ambiental.

A educação do intelecto requer que investiguemos as causas e condições
que subjazem a um futuro sustentável da raça humana. Tal envolve a
compreensão das consequências dos mais insignificantes actos de
destruição do planeta no sua globalidade, o que significa desistir do
nosso caminho habitual.

\section{Existe alguma abordagem budista para a resolução dos conflitos?}

Cada sociedade vive conflitos de interesse e de pontos de vista, quer
internos, quer externos. Os ensinamentos budistas enfatizam, antes de
mais, as formas de prevenir os conflitos, e de prevenir a escalada
daqueles que já tenham começado. Procuram alcançar isto instruindo as
pessoas envolvidas sobre a melhor forma de educarem a conduta, as
emoções e de compreenderem a vida.

No Budismo, a violência é considerada a reacção menos inteligente num
conflito. A violência, quer seja física ou verbal, não cria soluções
duradouras aos problemas. Os perpetuadores da violência criam carma
pesado com as suas acções, que acabarão por vir a pagar. As vítimas da
violência, ou suas famílias, anseiam por vingança. Os ciclos de
violência são postos em marcha. A raiz das causas do conflito fica por
sarar.

O Buda disse que as mentes livres de estados mentais nocivos tomam as
decisões mais inteligentes a longo prazo. A ganância, a importância
pessoal e o preconceito surgem nas mentes pessoais e, caso não se lhes
preste a devida atenção, podem ter consequências enormes para as
comunidades e para as nações. O Buda ensinou os discípulos a olharem
constantemente para o seu interior de maneira a perceberem com que
formas - acções e palavras, desejos e emoções, crenças, valores e
teorias - contribuem para os conflitos externos. Ensinou maneiras de
libertar os aspectos destrutivos da mente humana, e de alimentar os
construtivos. Ao aprenderem a distinguir as causas e as condições dos
conflitos, os budistas são ensinados a esforçarem"-se no sentido de
lidarem com isso da melhor forma possível.

\section{Qual a melhor forma de lidar com o stress?}

Dadas as nossas responsabilidades e as pressões exercidas sobre nós, é
inevitável que sintamos uma forte dose de stress. Nem sempre é mau, e é
mesmo difícil imaginar como é que fazer mudanças positivas na vida,
abandonar velhos maus hábitos, poderiam ser possíveis sem ter de se
passar por qualquer tipo de stress. Se não formos capazes de lidar com o
stress, ou se sentirmos que não deveríamos ter de lidar com ele,
poderemos não conseguir atingir importantes objectivos na vida.

Não obstante, é possível reduzir drasticamente a dose de stress com que
vivemos. Ajuda, se conseguirmos simplificar a vida tanto quanto
pudermos, e se aprendermos a abrandar um pouco; ao tentarmos ajustar num
dia demasiadas coisas, ganha"-se um cansaço desnecessário. Ao se dar
atenção à qualidade das nossas acções e palavras, reduzem"-se as
interacções stressantes com os demais. Um uso mais adequado dos recursos
reduz o stress nas finanças pessoais. Exercício regular, especialmente
yoga e tai chi (criados para actuar no sistema nervoso), alivia muito a
tensão física e ensina"-nos a respirar mais naturalmente.

A prática regular de meditação dá"-nos a capacidade de reconhecer e
largar os estados de espírito e os pensamentos prejudiciais que subjazem
à tensão crónica. As expectativas irrealistas sobre nós próprios e sobre
os que nos rodeiam, por exemplo, podemser incapacitantes. Na rotina
diária, criar breves intervalos para acalmar e centrar a mente evita que
o stress se acumule ao longo do dia. Estes intervalos podem tomar a
forma de uns breves sessenta segundos de meditação centrada na
respiração, em frente a um computador, ou de simplesmente respirar
profundamente algumas vezes antes de atender um telefone que toca.
Conseguir desenvolver a capacidade de regressar ao momento presente e
restabelecer um estado de calma alerta com regularidade ao longo do dia,
pode fazer a diferença de forma significativa na qualidade das nossas
vidas.

\section{Quão importante é ajudar os outros, no Budismo?}

O altruísmo está no centro do verdadeiro coração da tradição budista. O
brotar da compaixão é visto como a medida de uma mente desperta. As duas
virtudes inatas que se destacavam no Buda eram a sabedoria e a
compaixão. O Buda iluminou"-se através da sabedoria e partilhou o seu
caminho de despertar com os outros, pela compaixão. No Budismo, a
sabedoria e a compaixão eram consideradas inseparáveis, como duas asas
de uma águia.

O desejo de fazer os outros felizes ou de os aliviar do sofrimento é uma
maravilhosa jóia da mente humana. Mas, para que os nobres sentimentos
conduzam a uma acção efectiva, é preciso sabedoria. As pessoas com boa
intenção, mas com ausência de sensibilidade ou respeito por aqueles que
estão determinados em ajudar, podem fazer mais mal do que bem.
Preconceito, impaciência, hesitação, proferir a palavra errada, ou a
palavra certa no momento errado -- há tantos lapsos de julgamento,
tantas falhas de carácter que podem minar os melhores esforços da pessoa
mais bondosa. Além do mais, e talvez o mais importante, por vezes as
pessoas não querem ser ajudadas ou não estão preparadas para tal.

O Buda ensinou que a maturação é tudo. A pessoa sábia compreende que
todos os seres são `donos de seu próprio Karma'. Ao expressarem a
compaixão que brota da sabedoria e é ditada por ela, os sábios tentam
ajudar outros quando podem, nunca esquecendo que nada garante que os
seus esforços tenham sucesso. Como resultado, nunca se permitem
desapontar, nem desesperar, quando as coisas não resultam bem. Se se
frustram no esforço feito para ajudar os outros permanecem equânimes,
prontos para tentar de novo, sempre que as condições adequadas o
permitam.

\section{Qual é a atitude budista para com o vegetarianismo?}

O primeiro preceito assumido pelos budistas leigos requer que se
abstenham de matar seres vivos, ou que haja alguém, seu intermediário,
que mate seres vivos. Todas as formas intencionais de matar criam mau
carma, com sérias consequências para quem o faz. Contudo, no caso das
pessoas que compram e consomem carne de um supermercado, por exemplo,
não decorre qualquer carma, dado não estarem directamente envolvidos na
morte desse animal em particular que lhes fornece a carne. Não obstante,
muitos budistas adoptam uma dieta vegetariana, como uma vontade de
evitarem qualquer conexão indirecta com a morte dos animais.

O Buda também ensinou que as pessoas deviam reflectir na sua relação com
o ambiente onde vivem. Por esta razão, reduzir o consumo de carne, ou
adoptar uma dieta vegetariana pode ser considerado uma resposta
inteligente e budista, face à grave ameaça para com o ambiente, feita
pela crescente exigência do consumo de carne.
