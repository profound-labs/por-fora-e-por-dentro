\chapterPhotoTwoPageLeft{praying-Gr-crop}

\chapterPhotoTwoPageRight{praying-Gr-crop}

\setlength{\chapterTitleTopSkip}{60mm}
\definecolor{photoChapterText}{gray}{1}

\chapterNote{%
  É da minha natureza envelhecer; não estou para além do envelhecimento.

  É da minha natureza adoecer; não estou para além da doença.

  É da minha natureza morrer; não estou para além da morte.

  O que é meu, o que amo e prezo, mudará,
  separar"-se-á de mim.

  Sou dono das minhas acções, herdeiro das minhas acções, nascido das minhas
  acções, relacionado com as minhas acções, moldado pelas minhas acções. Serei o
  herdeiro de todas as acções que venha a praticar,\\ para bem e para mal.

  Assim, deveríamos lembrar"-nos disto frequentemente.

  Aṅguttara Nikāya, Livro dos Cinco 57
}

\enlargethispage{2\baselineskip}

\chapter{O Budismo na Tailândia}

\section{Existem muitas escolas de Budismo: qual é o tipo de Budismo praticado
  na Tailândia?}

Theravāda, o `caminho' (vāda) dos anciãos, é o nome da escola de
Budismo na Tailândia. É a forma de Budismo que se difundiu para sul a
partir do `País Central' no noroeste da Índia, floresceu no Sri Lanka
e depois espalhou"-se ao longo do Oceano Índico para o sul da Ásia.
Actualmente, aparte da sua presença na Tailândia, é o Budismo que existe
no Sri Lanka, Birmânia, Camboja, Laos, e em algumas partes do sul do Vietname.
O Budismo Theravāda é uma tradição conservadora, que se distingue
pela visão de que o corpo de ensinamentos, dados pelo Buda ao longo da
sua vida, está completo sem qualquer necessidade de correcção ou de
embelezamento. No Budismo Theravāda a principal tarefa é assegurar que
os ensinamentos de Buda, contidos no Canon Pāli (Tipițaka), sejam
preservados, estudados e postos em prática, de forma a que a verdade que
contêm possa ser vivida e, sempre que seja apropriado, partilhada com os
outros.

\section{Como é que o Budismo chegou à Tailândia?}

Aproximadamente duzentos anos depois do Buda ter falecido, o notório
imperador Asoka determinou que um pequeno grupo de monges viajasse pelo
mundo, partilhando o Dhamma com quem estivesse interessado em
aprendê"-lo. Um desses grupos foi enviado para o sudeste da Ásia, região
conhecida como Suvannabhumi, uma área que incluía o que hoje é a parte
central da Tailândia. Crê"-se que foi estabelecido um mosteiro no local
da moderna cidade Nakhon Phathom (mais tarde celebrado com um grandioso
stupa). Este é o primeiro relato do Budismo na Tailândia, embora seja
escassa a evidência histórica de tal. Existem, contudo, muitas
evidências arqueológicas que apontam para a importância do Budismo algum
tempo mais tarde, no período Dvāravati (do século VI ao séc.XII). É
provável que a civilização Dvāravati tenha sucumbido ao império de
Angkor e o Budismo Theravāda tenha sido largamente suplantado pelo
Bramanismo e mais tarde pelo Budismo Mahāyāna. A tradição Theravāda
restabeleceu"-se com a emergência do reino independente da Tailândia de
Sukhothai no séc. XIII. O primeiro rei de Sukhothai construiu um
mosteiro para a comunidade dos monges da floresta treinados no Sri
Lanka, que nessa altura tinham estado a viver no sul do país em Nakohn
Si Thammarat. Isto assinalou o começo de uma relação estreita entre a
nação tailandesa e o Budismo Theravāda que se prolongou até à presente
data.

\section{Para um visitante, a prática religiosa dos budistas tailandeses nem
  sempre está de acordo com as escrituras clássicas do Budismo. Porque é que
  existe tal disparidade?}

O Budismo não é uma religião de cruzadas, nem tentou sequer alguma vez
converter ou eliminar os seus rivais; pelo contrário desejou viver com
eles em paz. Ao longo dos séculos, inclui"-se o facto de as comunidades
tailandesas terem influências conciliadoras com o Bramanismo indiano e
com origens chinesas, bem como com práticas animistas antigas. Onde quer
que e quando esta atitude tolerante não foi acompanhada por uma
transmissão precisa de ensinamentos budistas, as fronteiras entre as
diferentes tradições acabaram por tornar"-se confusas. Desta forma, um
número de crenças não"-budistas acabaram por deslizar para a corrente
budista.

O derradeiro desafio enfrentado tem sido as enormes mudanças a nível
social e cultural trazidas pelo moderno desenvolvimento económico. Os
valores do mundo insinuaram"-se em muitas partes da comunidade budista.
Alguns mosteiros enriqueceram e não utilizaram a sua riqueza de forma
sábia. Ao mesmo tempo, uma reacção a este materialismo galopante
manifesta"-se de forma crescente e oferece esperança para o futuro.

\section{Parece que os budistas tailandeses fazem muitas vénias. Porque é que se
  curvam perante as estátuas do Buda?}

O Buda foi Aquele que despertou. A sua mente foi liberta de todas as
aflições e alcançou a perfeição na sabedoria, compaixão, pureza interior
e paz. Mas todas estas virtudes -- a essência do estado de ser Buda e o
objecto da devoção budista -- são qualidades abstractas, e a maioria das
pessoas precisa de um foco visível para a sua reverência e memória. As
estátuas de Buda oferecem esse foco.

Os budistas criaram as suas primeiras estátuas inspirados nas de Apolo
erigidas na colónia grega de Gandhara (uma área que cobre partes do
actual Paquistão e Afeganistão). As estátuas de Buda não pretendem ser
representações realistas do Buda histórico, sendo antes figuras que
evocam as qualidades inspiradoras que tornaram o Buda único. Curvar"-se
perante o Buda, primeiro de tudo, é um acto de devoção a uma forma que
representa `O Completamente Iluminado', `o professor inultrapassável
de deuses e de humanos', ou, como é frequentemente chamado, `o
grandioso médico'. Também é uma forma de humildade de quem se curva --
expressa pelo toque da cabeça no chão -- perante as virtudes do Buda e
um relembrar do seu compromisso para cultivar essas virtudes.

Os budistas fazem a vénia por três vezes à estátua do Buda. A segunda
vénia é ao Dhamma, a verdade dos ensinamentos do Buda que levaram à
realização dessa verdade. A terceira vénia é ao Sangha, a comunidade dos
seus discípulos iluminados.

\section{Os cânticos são uma espécie de orações?}

Uma oração é geralmente assumida por envolver uma relação com uma
divindade, o que não descreveria de forma justa a prática do canto
budista. A compreensão da acção e dos resultados no Budismo Theravāda
não dá espaço a orações de súplica ou de agradecimento. Todavia, existe
alguma similitude entre os cânticos que louvam as qualidades do Buda, do
Dhamma e do Sangha, e os hinos de louvor encontrados nas tradições
teístas. Muitas pessoas acreditam que um poder protector e auspicioso
surge no coração por cantar tais versos.

\section{Qual o valor dos cânticos?}

A maioria dos cânticos mais populares encontrados na tradição budista
tailandesa consiste nas passagens seleccionadas do Tipițaka. Incluem
versos que enunciam as qualidades do Buda, do Dhamma e do Sangha,
discursos que explanam ensinamentos"-chave, passagens de sábias
reflexões, versos para irradiarem pensamentos de bondade e partilha
de méritos com todos os seres vivos.

Para muitos budistas leigos tailandeses cantar é a sua principal prática
espiritual. Serve, especialmente, àqueles com uma disposição mais
activa, que encontram muita dificuldade nas práticas de meditação
sentada. Algumas pessoas preferem cânticos na língua original Pāli,
mesmo sem compreenderem o significado, como um acto de devoção e pelo
efeito meditativo da calma que oferece. Mas hoje em dia, é vulgar
cantar"-se no estilo moderno, onde cada linha do Pāli é seguida pela
tradução em Tailandês. Neste caso os benefícios viram"-se mais para o
relembrar do significado dos textos cantados.

Nos mosteiros, cantar discursos importantes é uma prática que remonta ao
tempo de Buda. Antes dos ensinamentos terem sido escritos, eram
preservados pelas comunidades de monges que, em conjunto, os cantavam
com regularidade. O acto de cantar também desempenha uma função social
nos mosteiros, onde as sessões de cânticos matutinos e vespertinos
ajudam a criar um sentido de comunidade e harmonia.

\section{Qual o contributo dos mosteiros para a sociedade?}

Faz parte das intenções das comunidades monásticas providenciarem
orientação moral, intelectual e espiritual às comunidades leigas que as
apoiam. Proporcionam aos budistas leigos a oportunidade de obter mérito
por oferecerem apoio material à ordem monástica, e de dedicarem esse
mérito aos que já partiram. Orientam rituais funerários e
cremações. Cantam versos de bênçãos em momentos importantes da vida das
famílias.

Existem dois tipos de mosteiros: os que estão nas florestas e os que se
encontram nas vilas e centros urbanos. Isto reflecte a divisão das
ordens monásticas entre aqueles cujas vidas estão devotadas
principalmente à meditação e os que se devotam maioritariamente ao estudo,
e a outros deveres `clericais'.

A presença de um mosteiro na floresta tem normalmente um efeito de
elevação nas populações locais. Os monges da floresta levam uma vida de
rigor e muitos alcançam elevados estados espirituais que são uma
inspiração para os leigos apoiantes. Os budistas leigos vão ao mosteiro
de manhã para oferecerem comida e apoio material aos monges. Uma vez
ali, podem pôr questões ou receber ensinamentos do abade. A maioria dos
mosteiros oferece instalação grátis para leigos, homens e mulheres, que
procuram um período de retiro onde possam praticar meditação. Existe um
número crescente de mosteiros que, ao longo do ano, organizam retiros de
meditação para os apoiantes leigos.

A relação entre os mosteiros nas vilas e cidades, e as comunidades
locais, tem tendência a ser mais próxima do que no caso dos mosteiros
mais remotos na floresta. Antigamente, estes mosteiros desempenhavam um
papel multifacetado na sociedade tailandesa, incluindo o de centros
sociais, hospitais, escolas e hotéis. Antes de muitas destas funções
terem passado para o estado, os mosteiros eram o verdadeiro centro da
vida nas aldeias. Anda hoje, consideram"-se os três pilares da comunidade
rural: o conselho da aldeia, a sua escola e o seu mosteiro.

\section{É permitido aos monges budistas envolverem"-se na política?}

Os monges budistas renunciam a toda e qualquer actividade política
quando abandonam o mundo. Se os monges se envolvessem na política, tal
teria efeitos nefastos na sua paz mental, seria uma causa desnecessária
e mundana de conflito no interior das comunidades monásticas, e
colocaria em risco o único papel do Sangha na sociedade.

O Buda queria que a ordem monástica se mantivesse distante dos assuntos
políticos de forma a manter o seu papel de refúgio para os budistas de
qualquer convicção política. Um Sangha apartidário pode fornecer uma
presença de ligação e de conciliação na sociedade, e este é um papel que
tem sido bem desempenhado na Tailândia ao longo dos séculos. Se o
Sangha, no seu todo, se identificasse com algum partido político ou
programa em especial, os budistas leigos que estivessem na oposição
desses partido sentir"-se-iam alienados do mosteiro e, potencialmente, da
própria religião budista em si mesma o estaria também. Se um Sangha
politicamente activo apoiasse o lado perdedor numa luta política, poderia
ser perseguido, trazendo sérias consequências à sobrevivência a longa
prazo do corpo monástico.

Dos monásticos budistas, espera"-se orientação moral e espiritual a
oferecer à sociedade. Se os programas políticos entram em conflito com
os princípios budistas, é legítimo que os monásticos falem da importância
de se preservarem esses princípios sem se referirem a partidos
políticos ou a indivíduos pelo seu nome.

\section{Existe alguma diferença entre um templo e um mosteiro?}

Em Tailandês só existe uma palavra: `wat'. Os primeiros eruditos que
traduziram o Tailandês para Inglês adoptaram uma convenção em que os
`wat'es na floresta seriam referidos como `mosteiros', e os que
estavam nas áreas urbanas seriam os `templos'. A razão para tal
distinção era devida a ideias não budistas sobre o que era, ou não era,
um mosteiro, mais do que a qualquer diferença fundamental entre os dois
tipos de `wat'es.

Não obstante, existem casos referentes a determinados `wat', que não
possuem uma comunidade monástica residente e que são chamados de
`templos'. Embora tais `wat' sejam extremamente raros, um exemplo
familiar à maioria dos visitantes da Tailândia é o Templo do Buda de
Esmeralda, em Banguecoque.

\section{Quais os benefícios que advêm de visitar um mosteiro?}

Idealmente, um templo budista, ou mosteiro, é um lugar onde,
temporariamente, os budistas leigos podem pôr de lado as suas
preocupações e aflições, desejos mundanos e medos. É um lugar onde se
espera encontrar tranquilidade, beleza e bondade. Também é um espaço
onde se podem encontrar amigos com semelhantes formas de pensar,
realizar acções de mérito e experimentar as alegrias da doação e do
serviço. É um lugar onde se pode receber inspiração e reflexões sábias
de monges seniores. Os mosteiros são, também, lugares onde os budistas
leigos podem participar em cerimónias que marcam os acontecimentos mais
importantes de suas vidas: nascimentos, casamentos e falecimentos.

É claro que os mosteiros variam muito no que respeita à forma como
vivem este ideal. A atmosfera nos mosteiros situados nas áreas urbanas é
muito diferente da que se encontra nas florestas e montanhas. Em países
como a Tailândia, os budistas leigos são afortunados por poderem
escolher o tipo de mosteiro que se ajusta às suas necessidades.

\section{Os cinco preceitos são considerados o código de base moral para os
  leigos budistas. Porque é que, das pessoas que se consideram budistas, parece
  haver tão poucas a cumprir estes preceitos?}

Infelizmente, parece que muitos budistas leigos não consideram que a sua
conduta moral seja uma condição para se identificarem como tal.

O Budismo rejeita os ensinamentos morais baseados no estímulo da
compensação, sendo favorável a uma educação da conduta. Infelizmente,
quando a natureza da educação não se enraíza profundamente, os budistas
leigos podem tornar"-se mais omissos do que aqueles inflamados com o
desejo da recompensa divina e o medo do tormento eterno.

\section{Qual é o estado actual do Budismo Tailandês?}

É difícil avaliar a saúde do Budismo tailandês. Evidências abundantes de
corrupção e declínio, coexistem com crescentes sinais de uma
revitalização.

Efectivamente, o Budismo tailandês enfrenta algumas mudanças difíceis. A
ordem monástica não se encontra no seu melhor estado de saúde. É de
consenso geral que o seu sistema administrativo e a transmissão de
educação precisam de reforma. A adesão à disciplina monástica é,
frequentemente, parca. Poucos monges cumprem a regra da proibição de
aceitarem presentes em dinheiro. Até a economia baseada no dinheiro ter
ganho força na Tailândia, há cerca de cinquenta anos, tal não era um
problema sério. Mas a sociedade enriqueceu cada vez mais e os donativos
também foram aumentando. Os monges são confrontados com tentações muito
sérias e muitos sucumbem. Em vez de fazerem uma séria crítica ao
materialismo e aos valores consumistas, alguns mosteiros aderiram a
eles. Nas zonas rurais há muitos mosteiros vazios. Com as reduções
drásticas das taxas de natalidade e o apelo às cidades, há menos pessoas
a aderirem às ordens. Um número aproximado de 300 mil monges parece ser
muito, mas tem"-se mantido estável durante muitos anos, enquanto, nesse
mesmo período, a população geral duplicou. Historicamente, o bem"-estar
do Budismo esteve sempre ligado ao bem"-estar do Sangha. Por esta razão,
existem causas muito sérias de preocupação.

Na sociedade em geral, os valores consumistas espalharam os seus
tentáculos de forma ainda mais abrangente. Números enormes de jovens e
de pobres deixam os seus lares à procura de trabalho em Banguecoque e no
estrangeiro. Ao trabalharem longas horas em fábricas, longe do apoio da
família e do mosteiro, facilmente se alienam dos valores budistas. A
vida nas cidades é atarefada e stressante para a maioria deles.

Felizmente, existem muitos sinais de encorajamento. O interesse pela
meditação está em `alta'. Os mosteiros e os centros de meditação, que
oferecem retiros aos budistas leigos, estão a florescer. Todos os anos,
são vendidos e distribuídos gratuitamente imensos livros e de DVDs sobre
Budismo. Ao longo dos últimos anos, as estações de rádio budistas têm"-se
instalado em cada distrito, frequentemente orientadas por mosteiros, e
são muito populares. Um número substancial de pessoas, com falta de
tempo para frequentarem os mosteiros, envolvem"-se em fóruns online,
debatendo o Dhamma, partilhando ensinamentos que as inspiraram. É
particularmente encorajador o número de jovens a regressarem aos
ensinamentos e às práticas budistas, vendo"-as como uma inspiração nas
suas vidas.

\section{O Budismo tem algum papel no sistema educativo tailandês?}

Sim, tem. Um grande número de escolas públicas na Tailândia usa o
`método Budista' (\emph{withee Bud}), embora não haja ainda um
consenso real sobre o que significa verdadeiramente esse termo. A
dimensão budista destas escolas varia bastante, e é muito determinada
pelas ideias dos seus funcionários. Um dos desenvolvimentos mais
interessantes nos últimos anos, tem sido um pequeno número de `escolas
budistas de sabedoria'. Nestas escolas os esforços são feitos para
adaptar, na vida da escola, os princípios de desenvolvimento imbuídos no
Óctuplo Caminho de Buda, não só a nível do currículo, mas também nas
relações entre professores, alunos e pais. No sistema holístico
visualizado nestas escolas, a educação é concebida como tendo quatro
dimensões, nomeadamente educação da:

\begin{enumerate}
\item Relação da criança com o mundo material;
\item Relação da criança com o mundo social;
\item Capacidade da criança para lidar sabiamente com os estados mentais
  prejudiciais e cultivar estados mentais elevados;
\item Capacidade da criança para pensar correctamente e reflecti"-la na
  experiência;
\end{enumerate}

Acrescido a isto, um número de mosteiros nas áreas urbanas organizam as
escolas de Domingo, baseadas no modelo cristão.

\section{Quais são os feriados budistas principais?}

\enlargethispage{\baselineskip}

Na Tailândia, são celebrados três feriados budistas: Māgha Pūjā, Visākha
Pūjā e Asālha Pūjā. As datas destes feriados variam de ano para ano,
determinadas pelo calendário lunar, não pelo solar. Os feriados
comemoram acontecimentos importantes que se deram nos dias de lua cheia,
no tempo do Buda. Cada feriado é dedicado a um dos três refúgios: Māgha
Pūjā, ao Dhamma, Visākha Pūjā ao Buda e Asālha Pūjā, ao Sangha.

Māgha Pūjā é celebrado na lua cheia de Fevereiro. Celebra o dia em que o
Buda apresentou o discurso Ovāda Pātimokkha, no qual resumiu os
ensinamentos de todos os Budas. A ocasião é considerada especialmente
auspiciosa, uma vez que a audiência continha 1250 monges iluminados, que
se juntaram no mosteiro onde o Buda residia, sem combinação prévia.

Visākha Pūjā é celebrado na lua cheia de Maio. Acredita"-se que foi neste
dia que Buda nasceu, se iluminou e faleceu. É um dia dedicado à memória
de Buda e é considerada a data mais importante do calendário Budista.

Asālha Pūjā é celebrado na lua cheia de Julho. Comemora o dia em que o
Buda transmitiu o seu primeiro discurso, o Dhammacakkhapavatanna Sutta,
o qual `pôs em movimento a roda do Dhamma'. A audiência era composta
pelos seus cinco primeiros seguidores, que o tinham acompanhado durante
anos de práticas ascéticas. No fim do discurso, um destes ascetas, Aññā
Kondañña, atingiu o primeiro estado de iluminação, tornando"-se, assim, o
primeiro membro da `comunidade dos nobres', ou Sangha.

Nos feriados budistas, os leigos participam numa série de actividades
meritórias: oferecem comida à ordem monástica, tomam preceitos
renunciantes, ouvem sermões, meditam; mas a actividade mais popular é a
participação na circum"-ambulação à volta das imagens de Buda, ou stupas
com relicários, que muitos mosteiros organizam nessas noites, assim que
surge a lua cheia.

\section{Parece existir um significativo número de crimes e de corrupção na
  Tailândia. Como é que tal é possível num país totalmente budista?}

O crime e a corrupção são universais. A filiação religiosa, ou a falta
dela, é só um dos factores entre muitos que determinam o nível de crime
numa sociedade - a pobreza, por um lado, é um indicador mais fiável. Não
obstante, supondo"-se que havia uma relação entre os problemas
enfrentados numa sociedade e a sua religião dominante, essa relação
poderia basear"-se em:

\begin{enumerate}
\item Pessoas que justificam más acções com ensinamentos religiosos;
\item Pessoas que justificam más acções distorcendo os ensinamentos religiosos;
\item Pessoas que agem em directa oposição aos ensinamentos dessas religiões.
\end{enumerate}

Entre os tailandeses que se consideram budistas, 1) é desconhecido, 2) é
raro, 3) é comum.

Também se pode defender que o grau de crime e corrupção na Tailândia é
um indicativo do pouco que os seus líderes políticos fizeram para
assegurar que os valores budistas se mantivessem, num período de rápidas
mudanças sociais e económicas.

\section{Segundo parece os Tailandeses têm muito medo de fantasmas. Isto deve"-se
  aos ensinamentos budistas?}

Durante milhares de anos, os tailandeses foram animistas, antes de se
tornarem budistas. Como resultado disso, existe um sentido de imanência
do mundo invisível, profundamente incorporado na cultura tailandesa. Em
todas as épocas, um certo número de meditadores budistas desenvolvem a
capacidade de se aperceberem de seres de outros reinos. As suas
vivências garantem que, até no mundo moderno, a crença em fantasmas não
declina.

Os tailandeses sempre gostaram de histórias de fantasmas, e começam a
ouvi"-las na infância, numa idade facilmente impressionável. Com o avanço
dos efeitos especiais computorizados, filmes sofisticados e programas de
televisão continuam a manter o assunto dos fantasmas na vanguarda das
mentes humanas.

O Buda ensinou que relembrar as virtudes do Buda, do Dhamma e do Sangha,
ajuda a remover o medo da mente, qualquer que seja a sua causa. Ensinou
a desenvolver a plena atenção, a qual permite analisar o medo como sendo
simplesmente um estado mental condicionado, que surge e desaparece, de
acordo com as causas e as condições.

\section{Qual o objectivo das casas de espíritos que as pessoas colocam nos seus
  jardins?}

Os tailandeses sempre acreditaram que a maioria das áreas dos terrenos
são supervisionados por um espírito guardião, e todos os que constroem
algo devem, primeiro de tudo, pedir permissão ao espírito, e demostrar
respeito por ele, sempre. Claro que nem todos acreditam nisto, mas até
os que não têm essa tendência, consideram que `mais vale prevenir do
que remediar', seguindo, assim, a antiga tradição de colocar uma
pequena casa de espíritos num local apropriado nos seus terrenos.
